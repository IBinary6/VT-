\documentclass[]{article}
\usepackage{lmodern}
\usepackage{amssymb,amsmath}
\usepackage{ifxetex,ifluatex}
\usepackage[landscape]{geometry}
\usepackage{fixltx2e} % provides \textsubscript
\ifnum 0\ifxetex 1\fi\ifluatex 1\fi=0 % if pdftex
  \usepackage[T1]{fontenc}
  \usepackage[utf8]{inputenc}
\else % if luatex or xelatex
  \ifxetex
    \usepackage{mathspec}
  \else
    \usepackage{fontspec}
  \fi
  \defaultfontfeatures{Ligatures=TeX,Scale=MatchLowercase}
\fi
% use upquote if available, for straight quotes in verbatim environments
\IfFileExists{upquote.sty}{\usepackage{upquote}}{}
% use microtype if available
\IfFileExists{microtype.sty}{%
\usepackage[]{microtype}
\UseMicrotypeSet[protrusion]{basicmath} % disable protrusion for tt fonts
}{}
\PassOptionsToPackage{hyphens}{url} % url is loaded by hyperref
\usepackage[unicode=true]{hyperref}
\hypersetup{
            pdftitle={proc(5) - Linux manual page},
            pdfborder={0 0 0},
            breaklinks=true}
\urlstyle{same}  % don't use monospace font for urls
\usepackage{longtable,booktabs}
% Fix footnotes in tables (requires footnote package)
\IfFileExists{footnote.sty}{\usepackage{footnote}\makesavenoteenv{long table}}{}
\usepackage{graphicx,grffile}
\makeatletter
\def\maxwidth{\ifdim\Gin@nat@width>\linewidth\linewidth\else\Gin@nat@width\fi}
\def\maxheight{\ifdim\Gin@nat@height>\textheight\textheight\else\Gin@nat@height\fi}
\makeatother
% Scale images if necessary, so that they will not overflow the page
% margins by default, and it is still possible to overwrite the defaults
% using explicit options in \includegraphics[width, height, ...]{}
\setkeys{Gin}{width=\maxwidth,height=\maxheight,keepaspectratio}
\IfFileExists{parskip.sty}{%
\usepackage{parskip}
}{% else
\setlength{\parindent}{0pt}
\setlength{\parskip}{6pt plus 2pt minus 1pt}
}
\setlength{\emergencystretch}{3em}  % prevent overfull lines
\providecommand{\tightlist}{%
  \setlength{\itemsep}{0pt}\setlength{\parskip}{0pt}}
\setcounter{secnumdepth}{0}
% Redefines (sub)paragraphs to behave more like sections
\ifx\paragraph\undefined\else
\let\oldparagraph\paragraph
\renewcommand{\paragraph}[1]{\oldparagraph{#1}\mbox{}}
\fi
\ifx\subparagraph\undefined\else
\let\oldsubparagraph\subparagraph
\renewcommand{\subparagraph}[1]{\oldsubparagraph{#1}\mbox{}}
\fi

% set default figure placement to htbp
\makeatletter
\def\fps@figure{htbp}
\makeatother


\title{proc(5) - Linux manual page}
\date{}

\begin{document}
\maketitle

\protect\hypertarget{top_of_page}{}{}

\begin{longtable}[]{@{}ll@{}}
\toprule
\begin{minipage}[t]{0.47\columnwidth}\raggedright\strut
\href{../../../index.html}{man7.org} \textgreater{} Linux \textgreater{}
\href{../index.html}{man-pages}\strut
\end{minipage} & \begin{minipage}[t]{0.47\columnwidth}\raggedright\strut
\href{http://man7.org/training/}{Linux/UNIX system programming
training}\strut
\end{minipage}\tabularnewline
\bottomrule
\end{longtable}

\begin{center}\rule{0.5\linewidth}{\linethickness}\end{center}

\begin{longtable}[]{@{}lll@{}}
\toprule
\begin{minipage}[t]{0.32\columnwidth}\raggedright\strut
\protect\hyperlink{NAME}{NAME} \textbar{}
\protect\hyperlink{DESCRIPTION}{DESCRIPTION} \textbar{}
\protect\hyperlink{NOTES}{NOTES} \textbar{}
\protect\hyperlink{SEE_ALSO}{SEE~ALSO} \textbar{}
\protect\hyperlink{COLOPHON}{COLOPHON}\strut
\end{minipage} & \begin{minipage}[t]{0.32\columnwidth}\raggedright\strut
\strut
\end{minipage} & \begin{minipage}[t]{0.32\columnwidth}\raggedright\strut
\strut
\end{minipage}\tabularnewline
\bottomrule
\end{longtable}

\begin{verbatim}
PROC(5)                   Linux Programmer's Manual                  PROC(5)
\end{verbatim}

\subsection{\texorpdfstring{\protect\hyperlink{NAME}{}NAME ~ ~ ~ ~
\protect\hyperlink{top_of_page}{{top}}}{NAME ~ ~ ~ ~ top}}\label{name-top}

\begin{verbatim}
       proc - process information pseudo-filesystem
\end{verbatim}

\subsection{\texorpdfstring{\protect\hyperlink{DESCRIPTION}{}DESCRIPTION
~ ~ ~ ~
\protect\hyperlink{top_of_page}{{top}}}{DESCRIPTION ~ ~ ~ ~ top}}\label{description-top}

\begin{verbatim}
       The proc filesystem is a pseudo-filesystem which provides an
       interface to kernel data structures.  It is commonly mounted at
       /proc.  Typically, it is mounted automatically by the system, but it
       can also be mounted manually using a command such as:

           mount -t proc proc /proc

       Most of the files in the proc filesystem are read-only, but some
       files are writable, allowing kernel variables to be changed.

   Mount options
       The proc filesystem supports the following mount options:

       hidepid=n (since Linux 3.3)
              This option controls who can access the information in
              /proc/[pid] directories.  The argument, n, is one of the fol‐
              lowing values:

              0   Everybody may access all /proc/[pid] directories.  This is
                  the traditional behavior, and the default if this mount
                  option is not specified.

              1   Users may not access files and subdirectories inside any
                  /proc/[pid] directories but their own (the /proc/[pid]
                  directories themselves remain visible).  Sensitive files
                  such as /proc/[pid]/cmdline and /proc/[pid]/status are now
                  protected against other users.  This makes it impossible
                  to learn whether any user is running a specific program
                  (so long as the program doesn't otherwise reveal itself by
                  its behavior).

              2   As for mode 1, but in addition the /proc/[pid] directories
                  belonging to other users become invisible.  This means
                  that /proc/[pid] entries can no longer be used to discover
                  the PIDs on the system.  This doesn't hide the fact that a
                  process with a specific PID value exists (it can be
                  learned by other means, for example, by "kill -0 $PID"),
                  but it hides a process's UID and GID, which could other‐
                  wise be learned by employing stat(2) on a /proc/[pid]
                  directory.  This greatly complicates an attacker's task of
                  gathering information about running processes (e.g., dis‐
                  covering whether some daemon is running with elevated
                  privileges, whether another user is running some sensitive
                  program, whether other users are running any program at
                  all, and so on).

       gid=gid (since Linux 3.3)
              Specifies the ID of a group whose members are authorized to
              learn process information otherwise prohibited by hidepid
              (i.e., users in this group behave as though /proc was mounted
              with hidepid=0).  This group should be used instead of
              approaches such as putting nonroot users into the sudoers(5)
              file.

   Files and directories
       The following list describes many of the files and directories under
       the /proc hierarchy.

       /proc/[pid]
              There is a numerical subdirectory for each running process;
              the subdirectory is named by the process ID.

              Each /proc/[pid] subdirectory contains the pseudo-files and
              directories described below.  These files are normally owned
              by the effective user and effective group ID of the process.
              However, as a security measure, the ownership is made
              root:root if the process's "dumpable" attribute is set to a
              value other than 1.  This attribute may change for the follow‐
              ing reasons:

              *  The attribute was explicitly set via the prctl(2)
                 PR_SET_DUMPABLE operation.

              *  The attribute was reset to the value in the file
                 /proc/sys/fs/suid_dumpable (described below), for the rea‐
                 sons described in prctl(2).

              Resetting the "dumpable" attribute to 1 reverts the ownership
              of the /proc/[pid]/* files to the process's real UID and real
              GID.

       /proc/[pid]/attr
              The files in this directory provide an API for security mod‐
              ules.  The contents of this directory are files that can be
              read and written in order to set security-related attributes.
              This directory was added to support SELinux, but the intention
              was that the API be general enough to support other security
              modules.  For the purpose of explanation, examples of how
              SELinux uses these files are provided below.

              This directory is present only if the kernel was configured
              with CONFIG_SECURITY.

       /proc/[pid]/attr/current (since Linux 2.6.0)
              The contents of this file represent the current security
              attributes of the process.

              In SELinux, this file is used to get the security context of a
              process.  Prior to Linux 2.6.11, this file could not be used
              to set the security context (a write was always denied), since
              SELinux limited process security transitions to execve(2) (see
              the description of /proc/[pid]/attr/exec, below).  Since Linux
              2.6.11, SELinux lifted this restriction and began supporting
              "set" operations via writes to this node if authorized by pol‐
              icy, although use of this operation is only suitable for
              applications that are trusted to maintain any desired separa‐
              tion between the old and new security contexts.  Prior to
              Linux 2.6.28, SELinux did not allow threads within a multi-
              threaded process to set their security context via this node
              as it would yield an inconsistency among the security contexts
              of the threads sharing the same memory space.  Since Linux
              2.6.28, SELinux lifted this restriction and began supporting
              "set" operations for threads within a multithreaded process if
              the new security context is bounded by the old security con‐
              text, where the bounded relation is defined in policy and
              guarantees that the new security context has a subset of the
              permissions of the old security context.  Other security mod‐
              ules may choose to support "set" operations via writes to this
              node.

       /proc/[pid]/attr/exec (since Linux 2.6.0)
              This file represents the attributes to assign to the process
              upon a subsequent execve(2).

              In SELinux, this is needed to support role/domain transitions,
              and execve(2) is the preferred point to make such transitions
              because it offers better control over the initialization of
              the process in the new security label and the inheritance of
              state.  In SELinux, this attribute is reset on execve(2) so
              that the new program reverts to the default behavior for any
              execve(2) calls that it may make.  In SELinux, a process can
              set only its own /proc/[pid]/attr/exec attribute.

       /proc/[pid]/attr/fscreate (since Linux 2.6.0)
              This file represents the attributes to assign to files created
              by subsequent calls to open(2), mkdir(2), symlink(2), and
              mknod(2)

              SELinux employs this file to support creation of a file (using
              the aforementioned system calls) in a secure state, so that
              there is no risk of inappropriate access being obtained
              between the time of creation and the time that attributes are
              set.  In SELinux, this attribute is reset on execve(2), so
              that the new program reverts to the default behavior for any
              file creation calls it may make, but the attribute will per‐
              sist across multiple file creation calls within a program
              unless it is explicitly reset.  In SELinux, a process can set
              only its own /proc/[pid]/attr/fscreate attribute.

       /proc/[pid]/attr/keycreate (since Linux 2.6.18)
              If a process writes a security context into this file, all
              subsequently created keys (add_key(2)) will be labeled with
              this context.  For further information, see the kernel source
              file Documentation/security/keys/core.rst (or file Documenta‐
              tion/security/keys.txt on Linux between 3.0 and 4.13, or Docu‐
              mentation/keys.txt before Linux 3.0).

       /proc/[pid]/attr/prev (since Linux 2.6.0)
              This file contains the security context of the process before
              the last execve(2); that is, the previous value of
              /proc/[pid]/attr/current.

       /proc/[pid]/attr/socketcreate (since Linux 2.6.18)
              If a process writes a security context into this file, all
              subsequently created sockets will be labeled with this con‐
              text.

       /proc/[pid]/autogroup (since Linux 2.6.38)
              See sched(7).

       /proc/[pid]/auxv (since 2.6.0-test7)
              This contains the contents of the ELF interpreter information
              passed to the process at exec time.  The format is one
              unsigned long ID plus one unsigned long value for each entry.
              The last entry contains two zeros.  See also getauxval(3).

              Permission to access this file is governed by a ptrace access
              mode PTRACE_MODE_READ_FSCREDS check; see ptrace(2).

       /proc/[pid]/cgroup (since Linux 2.6.24)
              See cgroups(7).

       /proc/[pid]/clear_refs (since Linux 2.6.22)

              This is a write-only file, writable only by owner of the
              process.

              The following values may be written to the file:

              1 (since Linux 2.6.22)
                     Reset the PG_Referenced and ACCESSED/YOUNG bits for all
                     the pages associated with the process.  (Before kernel
                     2.6.32, writing any nonzero value to this file had this
                     effect.)

              2 (since Linux 2.6.32)
                     Reset the PG_Referenced and ACCESSED/YOUNG bits for all
                     anonymous pages associated with the process.

              3 (since Linux 2.6.32)
                     Reset the PG_Referenced and ACCESSED/YOUNG bits for all
                     file-mapped pages associated with the process.

              Clearing the PG_Referenced and ACCESSED/YOUNG bits provides a
              method to measure approximately how much memory a process is
              using.  One first inspects the values in the "Referenced"
              fields for the VMAs shown in /proc/[pid]/smaps to get an idea
              of the memory footprint of the process.  One then clears the
              PG_Referenced and ACCESSED/YOUNG bits and, after some measured
              time interval, once again inspects the values in the "Refer‐
              enced" fields to get an idea of the change in memory footprint
              of the process during the measured interval.  If one is inter‐
              ested only in inspecting the selected mapping types, then the
              value 2 or 3 can be used instead of 1.

              Further values can be written to affect different properties:

              4 (since Linux 3.11)
                     Clear the soft-dirty bit for all the pages associated
                     with the process.  This is used (in conjunction with
                     /proc/[pid]/pagemap) by the check-point restore system
                     to discover which pages of a process have been dirtied
                     since the file /proc/[pid]/clear_refs was written to.

              5 (since Linux 4.0)
                     Reset the peak resident set size ("high water mark") to
                     the process's current resident set size value.

              Writing any value to /proc/[pid]/clear_refs other than those
              listed above has no effect.

              The /proc/[pid]/clear_refs file is present only if the CON‐
              FIG_PROC_PAGE_MONITOR kernel configuration option is enabled.

       /proc/[pid]/cmdline
              This read-only file holds the complete command line for the
              process, unless the process is a zombie.  In the latter case,
              there is nothing in this file: that is, a read on this file
              will return 0 characters.  The command-line arguments appear
              in this file as a set of strings separated by null bytes
              ('\0'), with a further null byte after the last string.

       /proc/[pid]/comm (since Linux 2.6.33)
              This file exposes the process's comm value—that is, the com‐
              mand name associated with the process.  Different threads in
              the same process may have different comm values, accessible
              via /proc/[pid]/task/[tid]/comm.  A thread may modify its comm
              value, or that of any of other thread in the same thread group
              (see the discussion of CLONE_THREAD in clone(2)), by writing
              to the file /proc/self/task/[tid]/comm.  Strings longer than
              TASK_COMM_LEN (16) characters are silently truncated.

              This file provides a superset of the prctl(2) PR_SET_NAME and
              PR_GET_NAME operations, and is employed by
              pthread_setname_np(3) when used to rename threads other than
              the caller.

       /proc/[pid]/coredump_filter (since Linux 2.6.23)
              See core(5).

       /proc/[pid]/cpuset (since Linux 2.6.12)
              See cpuset(7).

       /proc/[pid]/cwd
              This is a symbolic link to the current working directory of
              the process.  To find out the current working directory of
              process 20, for instance, you can do this:

                  $ cd /proc/20/cwd; /bin/pwd

              Note that the pwd command is often a shell built-in, and might
              not work properly.  In bash(1), you may use pwd -P.

              In a multithreaded process, the contents of this symbolic link
              are not available if the main thread has already terminated
              (typically by calling pthread_exit(3)).

              Permission to dereference or read (readlink(2)) this symbolic
              link is governed by a ptrace access mode
              PTRACE_MODE_READ_FSCREDS check; see ptrace(2).

       /proc/[pid]/environ
              This file contains the initial environment that was set when
              the currently executing program was started via execve(2).
              The entries are separated by null bytes ('\0'), and there may
              be a null byte at the end.  Thus, to print out the environment
              of process 1, you would do:

                  $ strings /proc/1/environ

              If, after an execve(2), the process modifies its environment
              (e.g., by calling functions such as putenv(3) or modifying the
              environ(7) variable directly), this file will not reflect
              those changes.

              Furthermore, a process may change the memory location that
              this file refers via prctl(2) operations such as
              PR_SET_MM_ENV_START.

              Permission to access this file is governed by a ptrace access
              mode PTRACE_MODE_READ_FSCREDS check; see ptrace(2).

       /proc/[pid]/exe
              Under Linux 2.2 and later, this file is a symbolic link con‐
              taining the actual pathname of the executed command.  This
              symbolic link can be dereferenced normally; attempting to open
              it will open the executable.  You can even type
              /proc/[pid]/exe to run another copy of the same executable
              that is being run by process [pid].  If the pathname has been
              unlinked, the symbolic link will contain the string
              '(deleted)' appended to the original pathname.  In a multi‐
              threaded process, the contents of this symbolic link are not
              available if the main thread has already terminated (typically
              by calling pthread_exit(3)).

              Permission to dereference or read (readlink(2)) this symbolic
              link is governed by a ptrace access mode
              PTRACE_MODE_READ_FSCREDS check; see ptrace(2).

              Under Linux 2.0 and earlier, /proc/[pid]/exe is a pointer to
              the binary which was executed, and appears as a symbolic link.
              A readlink(2) call on this file under Linux 2.0 returns a
              string in the format:

                  [device]:inode

              For example, [0301]:1502 would be inode 1502 on device major
              03 (IDE, MFM, etc. drives) minor 01 (first partition on the
              first drive).

              find(1) with the -inum option can be used to locate the file.

       /proc/[pid]/fd/
              This is a subdirectory containing one entry for each file
              which the process has open, named by its file descriptor, and
              which is a symbolic link to the actual file.  Thus, 0 is stan‐
              dard input, 1 standard output, 2 standard error, and so on.

              For file descriptors for pipes and sockets, the entries will
              be symbolic links whose content is the file type with the
              inode.  A readlink(2) call on this file returns a string in
              the format:

                  type:[inode]

              For example, socket:[2248868] will be a socket and its inode
              is 2248868.  For sockets, that inode can be used to find more
              information in one of the files under /proc/net/.

              For file descriptors that have no corresponding inode (e.g.,
              file descriptors produced by bpf(2), epoll_create(2),
              eventfd(2), inotify_init(2), perf_event_open(2), signalfd(2),
              timerfd_create(2), and userfaultfd(2)), the entry will be a
              symbolic link with contents of the form

                  anon_inode:<file-type>

              In many cases (but not all), the file-type is surrounded by
              square brackets.

              For example, an epoll file descriptor will have a symbolic
              link whose content is the string anon_inode:[eventpoll].

              In a multithreaded process, the contents of this directory are
              not available if the main thread has already terminated (typi‐
              cally by calling pthread_exit(3)).

              Programs that take a filename as a command-line argument, but
              don't take input from standard input if no argument is sup‐
              plied, and programs that write to a file named as a command-
              line argument, but don't send their output to standard output
              if no argument is supplied, can nevertheless be made to use
              standard input or standard output by using /proc/[pid]/fd
              files as command-line arguments.  For example, assuming that
              -i is the flag designating an input file and -o is the flag
              designating an output file:

                  $ foobar -i /proc/self/fd/0 -o /proc/self/fd/1 ...

              and you have a working filter.

              /proc/self/fd/N is approximately the same as /dev/fd/N in some
              UNIX and UNIX-like systems.  Most Linux MAKEDEV scripts sym‐
              bolically link /dev/fd to /proc/self/fd, in fact.

              Most systems provide symbolic links /dev/stdin, /dev/stdout,
              and /dev/stderr, which respectively link to the files 0, 1,
              and 2 in /proc/self/fd.  Thus the example command above could
              be written as:

                  $ foobar -i /dev/stdin -o /dev/stdout ...

              Permission to dereference or read (readlink(2)) the symbolic
              links in this directory is governed by a ptrace access mode
              PTRACE_MODE_READ_FSCREDS check; see ptrace(2).

              Note that for file descriptors referring to inodes (pipes and
              sockets, see above), those inodes still have permission bits
              and ownership information distinct from those of the
              /proc/[pid]/fd entry, and that the owner may differ from the
              user and group IDs of the process.  An unprivileged process
              may lack permissions to open them, as in this example:

                  $ echo test | sudo -u nobody cat
                  test
                  $ echo test | sudo -u nobody cat /proc/self/fd/0
                  cat: /proc/self/fd/0: Permission denied

              File descriptor 0 refers to the pipe created by the shell and
              owned by that shell's user, which is not nobody, so cat does
              not have permission to create a new file descriptor to read
              from that inode, even though it can still read from its exist‐
              ing file descriptor 0.

       /proc/[pid]/fdinfo/ (since Linux 2.6.22)
              This is a subdirectory containing one entry for each file
              which the process has open, named by its file descriptor.  The
              files in this directory are readable only by the owner of the
              process.  The contents of each file can be read to obtain
              information about the corresponding file descriptor.  The con‐
              tent depends on the type of file referred to by the corre‐
              sponding file descriptor.

              For regular files and directories, we see something like:

                  $ cat /proc/12015/fdinfo/4
                  pos:    1000
                  flags:  01002002
                  mnt_id: 21

              The fields are as follows:

              pos    This is a decimal number showing the file offset.

              flags  This is an octal number that displays the file access
                     mode and file status flags (see open(2)).  If the
                     close-on-exec file descriptor flag is set, then flags
                     will also include the value O_CLOEXEC.

                     Before Linux 3.1, this field incorrectly displayed the
                     setting of O_CLOEXEC at the time the file was opened,
                     rather than the current setting of the close-on-exec
                     flag.

              mnt_id This field, present since Linux 3.15, is the ID of the
                     mount point containing this file.  See the description
                     of /proc/[pid]/mountinfo.

              For eventfd file descriptors (see eventfd(2)), we see (since
              Linux 3.8) the following fields:

                  pos: 0
                  flags:    02
                  mnt_id:   10
                  eventfd-count:               40

              eventfd-count is the current value of the eventfd counter, in
              hexadecimal.

              For epoll file descriptors (see epoll(7)), we see (since Linux
              3.8) the following fields:

                  pos: 0
                  flags:    02
                  mnt_id:   10
                  tfd:        9 events:       19 data: 74253d2500000009
                  tfd:        7 events:       19 data: 74253d2500000007

              Each of the lines beginning tfd describes one of the file
              descriptors being monitored via the epoll file descriptor (see
              epoll_ctl(2) for some details).  The tfd field is the number
              of the file descriptor.  The events field is a hexadecimal
              mask of the events being monitored for this file descriptor.
              The data field is the data value associated with this file
              descriptor.

              For signalfd file descriptors (see signalfd(2)), we see (since
              Linux 3.8) the following fields:

                  pos: 0
                  flags:    02
                  mnt_id:   10
                  sigmask:  0000000000000006

              sigmask is the hexadecimal mask of signals that are accepted
              via this signalfd file descriptor.  (In this example, bits 2
              and 3 are set, corresponding to the signals SIGINT and
              SIGQUIT; see signal(7).)

              For inotify file descriptors (see inotify(7)), we see (since
              Linux 3.8) the following fields:

                  pos: 0
                  flags:    00
                  mnt_id:   11
                  inotify wd:2 ino:7ef82a sdev:800001 mask:800afff ignored_mask:0 fhandle-bytes:8 fhandle-type:1 f_handle:2af87e00220ffd73
                  inotify wd:1 ino:192627 sdev:800001 mask:800afff ignored_mask:0 fhandle-bytes:8 fhandle-type:1 f_handle:27261900802dfd73

              Each of the lines beginning with "inotify" displays informa‐
              tion about one file or directory that is being monitored.  The
              fields in this line are as follows:

              wd     A watch descriptor number (in decimal).

              ino    The inode number of the target file (in hexadecimal).

              sdev   The ID of the device where the target file resides (in
                     hexadecimal).

              mask   The mask of events being monitored for the target file
                     (in hexadecimal).

              If the kernel was built with exportfs support, the path to the
              target file is exposed as a file handle, via three hexadecimal
              fields: fhandle-bytes, fhandle-type, and f_handle.

              For fanotify file descriptors (see fanotify(7)), we see (since
              Linux 3.8) the following fields:

                  pos: 0
                  flags:    02
                  mnt_id:   11
                  fanotify flags:0 event-flags:88002
                  fanotify ino:19264f sdev:800001 mflags:0 mask:1 ignored_mask:0 fhandle-bytes:8 fhandle-type:1 f_handle:4f261900a82dfd73

              The fourth line displays information defined when the fanotify
              group was created via fanotify_init(2):

              flags  The flags argument given to fanotify_init(2) (expressed
                     in hexadecimal).

              event-flags
                     The event_f_flags argument given to fanotify_init(2)
                     (expressed in hexadecimal).

              Each additional line shown in the file contains information
              about one of the marks in the fanotify group.  Most of these
              fields are as for inotify, except:

              mflags The flags associated with the mark (expressed in hexa‐
                     decimal).

              mask   The events mask for this mark (expressed in hexadeci‐
                     mal).

              ignored_mask
                     The mask of events that are ignored for this mark
                     (expressed in hexadecimal).

              For details on these fields, see fanotify_mark(2).

       /proc/[pid]/gid_map (since Linux 3.5)
              See user_namespaces(7).

       /proc/[pid]/io (since kernel 2.6.20)
              This file contains I/O statistics for the process, for exam‐
              ple:

                  # cat /proc/3828/io
                  rchar: 323934931
                  wchar: 323929600
                  syscr: 632687
                  syscw: 632675
                  read_bytes: 0
                  write_bytes: 323932160
                  cancelled_write_bytes: 0

              The fields are as follows:

              rchar: characters read
                     The number of bytes which this task has caused to be
                     read from storage.  This is simply the sum of bytes
                     which this process passed to read(2) and similar system
                     calls.  It includes things such as terminal I/O and is
                     unaffected by whether or not actual physical disk I/O
                     was required (the read might have been satisfied from
                     pagecache).

              wchar: characters written
                     The number of bytes which this task has caused, or
                     shall cause to be written to disk.  Similar caveats
                     apply here as with rchar.

              syscr: read syscalls
                     Attempt to count the number of read I/O operations—that
                     is, system calls such as read(2) and pread(2).

              syscw: write syscalls
                     Attempt to count the number of write I/O operations—
                     that is, system calls such as write(2) and pwrite(2).

              read_bytes: bytes read
                     Attempt to count the number of bytes which this process
                     really did cause to be fetched from the storage layer.
                     This is accurate for block-backed filesystems.

              write_bytes: bytes written
                     Attempt to count the number of bytes which this process
                     caused to be sent to the storage layer.

              cancelled_write_bytes:
                     The big inaccuracy here is truncate.  If a process
                     writes 1MB to a file and then deletes the file, it will
                     in fact perform no writeout.  But it will have been
                     accounted as having caused 1MB of write.  In other
                     words: this field represents the number of bytes which
                     this process caused to not happen, by truncating page‐
                     cache.  A task can cause "negative" I/O too.  If this
                     task truncates some dirty pagecache, some I/O which
                     another task has been accounted for (in its
                     write_bytes) will not be happening.

              Note: In the current implementation, things are a bit racy on
              32-bit systems: if process A reads process B's /proc/[pid]/io
              while process B is updating one of these 64-bit counters,
              process A could see an intermediate result.

              Permission to access this file is governed by a ptrace access
              mode PTRACE_MODE_READ_FSCREDS check; see ptrace(2).

       /proc/[pid]/limits (since Linux 2.6.24)
              This file displays the soft limit, hard limit, and units of
              measurement for each of the process's resource limits (see
              getrlimit(2)).  Up to and including Linux 2.6.35, this file is
              protected to allow reading only by the real UID of the
              process.  Since Linux 2.6.36, this file is readable by all
              users on the system.

       /proc/[pid]/map_files/ (since kernel 3.3)
              This subdirectory contains entries corresponding to memory-
              mapped files (see mmap(2)).  Entries are named by memory
              region start and end address pair (expressed as hexadecimal
              numbers), and are symbolic links to the mapped files them‐
              selves.  Here is an example, with the output wrapped and
              reformatted to fit on an 80-column display:

                  # ls -l /proc/self/map_files/
                  lr--------. 1 root root 64 Apr 16 21:31
                              3252e00000-3252e20000 -> /usr/lib64/ld-2.15.so
                  ...

              Although these entries are present for memory regions that
              were mapped with the MAP_FILE flag, the way anonymous shared
              memory (regions created with the MAP_ANON | MAP_SHARED flags)
              is implemented in Linux means that such regions also appear on
              this directory.  Here is an example where the target file is
              the deleted /dev/zero one:

                  lrw-------. 1 root root 64 Apr 16 21:33
                              7fc075d2f000-7fc075e6f000 -> /dev/zero (deleted)

              This directory appears only if the CONFIG_CHECKPOINT_RESTORE
              kernel configuration option is enabled.  Privilege
              (CAP_SYS_ADMIN) is required to view the contents of this
              directory.

       /proc/[pid]/maps
              A file containing the currently mapped memory regions and
              their access permissions.  See mmap(2) for some further infor‐
              mation about memory mappings.

              Permission to access this file is governed by a ptrace access
              mode PTRACE_MODE_READ_FSCREDS check; see ptrace(2).

              The format of the file is:

    address           perms offset  dev   inode       pathname
    00400000-00452000 r-xp 00000000 08:02 173521      /usr/bin/dbus-daemon
    00651000-00652000 r--p 00051000 08:02 173521      /usr/bin/dbus-daemon
    00652000-00655000 rw-p 00052000 08:02 173521      /usr/bin/dbus-daemon
    00e03000-00e24000 rw-p 00000000 00:00 0           [heap]
    00e24000-011f7000 rw-p 00000000 00:00 0           [heap]
    ...
    35b1800000-35b1820000 r-xp 00000000 08:02 135522  /usr/lib64/ld-2.15.so
    35b1a1f000-35b1a20000 r--p 0001f000 08:02 135522  /usr/lib64/ld-2.15.so
    35b1a20000-35b1a21000 rw-p 00020000 08:02 135522  /usr/lib64/ld-2.15.so
    35b1a21000-35b1a22000 rw-p 00000000 00:00 0
    35b1c00000-35b1dac000 r-xp 00000000 08:02 135870  /usr/lib64/libc-2.15.so
    35b1dac000-35b1fac000 ---p 001ac000 08:02 135870  /usr/lib64/libc-2.15.so
    35b1fac000-35b1fb0000 r--p 001ac000 08:02 135870  /usr/lib64/libc-2.15.so
    35b1fb0000-35b1fb2000 rw-p 001b0000 08:02 135870  /usr/lib64/libc-2.15.so
    ...
    f2c6ff8c000-7f2c7078c000 rw-p 00000000 00:00 0    [stack:986]
    ...
    7fffb2c0d000-7fffb2c2e000 rw-p 00000000 00:00 0   [stack]
    7fffb2d48000-7fffb2d49000 r-xp 00000000 00:00 0   [vdso]

              The address field is the address space in the process that the
              mapping occupies.  The perms field is a set of permissions:

                  r = read
                  w = write
                  x = execute
                  s = shared
                  p = private (copy on write)

              The offset field is the offset into the file/whatever; dev is
              the device (major:minor); inode is the inode on that device.
              0 indicates that no inode is associated with the memory
              region, as would be the case with BSS (uninitialized data).

              The pathname field will usually be the file that is backing
              the mapping.  For ELF files, you can easily coordinate with
              the offset field by looking at the Offset field in the ELF
              program headers (readelf -l).

              There are additional helpful pseudo-paths:

                   [stack]
                          The initial process's (also known as the main
                          thread's) stack.

                   [stack:<tid>] (since Linux 3.4)
                          A thread's stack (where the <tid> is a thread ID).
                          It corresponds to the /proc/[pid]/task/[tid]/
                          path.

                   [vdso] The virtual dynamically linked shared object.  See
                          vdso(7).

                   [heap] The process's heap.

              If the pathname field is blank, this is an anonymous mapping
              as obtained via mmap(2).  There is no easy way to coordinate
              this back to a process's source, short of running it through
              gdb(1), strace(1), or similar.

              Under Linux 2.0, there is no field giving pathname.

       /proc/[pid]/mem
              This file can be used to access the pages of a process's mem‐
              ory through open(2), read(2), and lseek(2).

              Permission to access this file is governed by a ptrace access
              mode PTRACE_MODE_ATTACH_FSCREDS check; see ptrace(2).

       /proc/[pid]/mountinfo (since Linux 2.6.26)
              This file contains information about mount points in the
              process's mount namespace (see mount_namespaces(7)).  It sup‐
              plies various information (e.g., propagation state, root of
              mount for bind mounts, identifier for each mount and its par‐
              ent) that is missing from the (older) /proc/[pid]/mounts file,
              and fixes various other problems with that file (e.g., nonex‐
              tensibility, failure to distinguish per-mount versus per-
              superblock options).

              The file contains lines of the form:

36 35 98:0 /mnt1 /mnt2 rw,noatime master:1 - ext3 /dev/root rw,errors=continue
(1)(2)(3)   (4)   (5)      (6)      (7)   (8) (9)   (10)         (11)

              The numbers in parentheses are labels for the descriptions
              below:

              (1)  mount ID: a unique ID for the mount (may be reused after
                   umount(2)).

              (2)  parent ID: the ID of the parent mount (or of self for the
                   root of this mount namespace's mount tree).

                   If the parent mount point lies outside the process's root
                   directory (see chroot(2)), the ID shown here won't have a
                   corresponding record in mountinfo whose mount ID (field
                   1) matches this parent mount ID (because mount points
                   that lie outside the process's root directory are not
                   shown in mountinfo).  As a special case of this point,
                   the process's root mount point may have a parent mount
                   (for the initramfs filesystem) that lies outside the
                   process's root directory, and an entry for that mount
                   point will not appear in mountinfo.

              (3)  major:minor: the value of st_dev for files on this
                   filesystem (see stat(2)).

              (4)  root: the pathname of the directory in the filesystem
                   which forms the root of this mount.

              (5)  mount point: the pathname of the mount point relative to
                   the process's root directory.

              (6)  mount options: per-mount options.

              (7)  optional fields: zero or more fields of the form
                   "tag[:value]"; see below.

              (8)  separator: the end of the optional fields is marked by a
                   single hyphen.

              (9)  filesystem type: the filesystem type in the form
                   "type[.subtype]".

              (10) mount source: filesystem-specific information or "none".

              (11) super options: per-superblock options.

              Currently, the possible optional fields are shared, master,
              propagate_from, and unbindable.  See mount_namespaces(7) for a
              description of these fields.  Parsers should ignore all unrec‐
              ognized optional fields.

              For more information on mount propagation see: Documenta‐
              tion/filesystems/sharedsubtree.txt in the Linux kernel source
              tree.

       /proc/[pid]/mounts (since Linux 2.4.19)
              This file lists all the filesystems currently mounted in the
              process's mount namespace (see mount_namespaces(7)).  The for‐
              mat of this file is documented in fstab(5).

              Since kernel version 2.6.15, this file is pollable: after
              opening the file for reading, a change in this file (i.e., a
              filesystem mount or unmount) causes select(2) to mark the file
              descriptor as having an exceptional condition, and poll(2) and
              epoll_wait(2) mark the file as having a priority event (POLL‐
              PRI).  (Before Linux 2.6.30, a change in this file was indi‐
              cated by the file descriptor being marked as readable for
              select(2), and being marked as having an error condition for
              poll(2) and epoll_wait(2).)

       /proc/[pid]/mountstats (since Linux 2.6.17)
              This file exports information (statistics, configuration
              information) about the mount points in the process's mount
              namespace (see mount_namespaces(7)).  Lines in this file have
              the form:

                  device /dev/sda7 mounted on /home with fstype ext3 [statistics]
                  (       1      )            ( 2 )             (3 ) (4)

              The fields in each line are:

              (1)  The name of the mounted device (or "nodevice" if there is
                   no corresponding device).

              (2)  The mount point within the filesystem tree.

              (3)  The filesystem type.

              (4)  Optional statistics and configuration information.  Cur‐
                   rently (as at Linux 2.6.26), only NFS filesystems export
                   information via this field.

              This file is readable only by the owner of the process.

       /proc/[pid]/net (since Linux 2.6.25)
              See the description of /proc/net.

       /proc/[pid]/ns/ (since Linux 3.0)
              This is a subdirectory containing one entry for each namespace
              that supports being manipulated by setns(2).  For more infor‐
              mation, see namespaces(7).

       /proc/[pid]/numa_maps (since Linux 2.6.14)
              See numa(7).

       /proc/[pid]/oom_adj (since Linux 2.6.11)
              This file can be used to adjust the score used to select which
              process should be killed in an out-of-memory (OOM) situation.
              The kernel uses this value for a bit-shift operation of the
              process's oom_score value: valid values are in the range -16
              to +15, plus the special value -17, which disables OOM-killing
              altogether for this process.  A positive score increases the
              likelihood of this process being killed by the OOM-killer; a
              negative score decreases the likelihood.

              The default value for this file is 0; a new process inherits
              its parent's oom_adj setting.  A process must be privileged
              (CAP_SYS_RESOURCE) to update this file.

              Since Linux 2.6.36, use of this file is deprecated in favor of
              /proc/[pid]/oom_score_adj.

       /proc/[pid]/oom_score (since Linux 2.6.11)
              This file displays the current score that the kernel gives to
              this process for the purpose of selecting a process for the
              OOM-killer.  A higher score means that the process is more
              likely to be selected by the OOM-killer.  The basis for this
              score is the amount of memory used by the process, with
              increases (+) or decreases (-) for factors including:

              * whether the process is privileged (-).

              Before kernel 2.6.36 the following factors were also used in
              the calculation of oom_score:

              * whether the process creates a lot of children using fork(2)
                (+);

              * whether the process has been running a long time, or has
                used a lot of CPU time (-);

              * whether the process has a low nice value (i.e., > 0) (+);
                and

              * whether the process is making direct hardware access (-).

              The oom_score also reflects the adjustment specified by the
              oom_score_adj or oom_adj setting for the process.

       /proc/[pid]/oom_score_adj (since Linux 2.6.36)
              This file can be used to adjust the badness heuristic used to
              select which process gets killed in out-of-memory conditions.

              The badness heuristic assigns a value to each candidate task
              ranging from 0 (never kill) to 1000 (always kill) to determine
              which process is targeted.  The units are roughly a proportion
              along that range of allowed memory the process may allocate
              from, based on an estimation of its current memory and swap
              use.  For example, if a task is using all allowed memory, its
              badness score will be 1000.  If it is using half of its
              allowed memory, its score will be 500.

              There is an additional factor included in the badness score:
              root processes are given 3% extra memory over other tasks.

              The amount of "allowed" memory depends on the context in which
              the OOM-killer was called.  If it is due to the memory
              assigned to the allocating task's cpuset being exhausted, the
              allowed memory represents the set of mems assigned to that
              cpuset (see cpuset(7)).  If it is due to a mempolicy's node(s)
              being exhausted, the allowed memory represents the set of mem‐
              policy nodes.  If it is due to a memory limit (or swap limit)
              being reached, the allowed memory is that configured limit.
              Finally, if it is due to the entire system being out of mem‐
              ory, the allowed memory represents all allocatable resources.

              The value of oom_score_adj is added to the badness score
              before it is used to determine which task to kill.  Acceptable
              values range from -1000 (OOM_SCORE_ADJ_MIN) to +1000
              (OOM_SCORE_ADJ_MAX).  This allows user space to control the
              preference for OOM-killing, ranging from always preferring a
              certain task or completely disabling it from OOM killing.  The
              lowest possible value, -1000, is equivalent to disabling OOM-
              killing entirely for that task, since it will always report a
              badness score of 0.

              Consequently, it is very simple for user space to define the
              amount of memory to consider for each task.  Setting an
              oom_score_adj value of +500, for example, is roughly equiva‐
              lent to allowing the remainder of tasks sharing the same sys‐
              tem, cpuset, mempolicy, or memory controller resources to use
              at least 50% more memory.  A value of -500, on the other hand,
              would be roughly equivalent to discounting 50% of the task's
              allowed memory from being considered as scoring against the
              task.

              For backward compatibility with previous kernels,
              /proc/[pid]/oom_adj can still be used to tune the badness
              score.  Its value is scaled linearly with oom_score_adj.

              Writing to /proc/[pid]/oom_score_adj or /proc/[pid]/oom_adj
              will change the other with its scaled value.

       /proc/[pid]/pagemap (since Linux 2.6.25)
              This file shows the mapping of each of the process's virtual
              pages into physical page frames or swap area.  It contains one
              64-bit value for each virtual page, with the bits set as fol‐
              lows:

                   63     If set, the page is present in RAM.

                   62     If set, the page is in swap space

                   61 (since Linux 3.5)
                          The page is a file-mapped page or a shared anony‐
                          mous page.

                   60–57 (since Linux 3.11)
                          Zero

                   56 (since Linux 4.2)
                          The page is exclusively mapped.

                   55 (since Linux 3.11)
                          PTE is soft-dirty (see the kernel source file Doc‐
                          umentation/vm/soft-dirty.txt).

                   54–0   If the page is present in RAM (bit 63), then these
                          bits provide the page frame number, which can be
                          used to index /proc/kpageflags and /proc/kpage‐
                          count.  If the page is present in swap (bit 62),
                          then bits 4–0 give the swap type, and bits 54–5
                          encode the swap offset.

              Before Linux 3.11, bits 60–55 were used to encode the base-2
              log of the page size.

              To employ /proc/[pid]/pagemap efficiently, use
              /proc/[pid]/maps to determine which areas of memory are actu‐
              ally mapped and seek to skip over unmapped regions.

              The /proc/[pid]/pagemap file is present only if the CON‐
              FIG_PROC_PAGE_MONITOR kernel configuration option is enabled.

              Permission to access this file is governed by a ptrace access
              mode PTRACE_MODE_READ_FSCREDS check; see ptrace(2).

       /proc/[pid]/personality (since Linux 2.6.28)
              This read-only file exposes the process's execution domain, as
              set by personality(2).  The value is displayed in hexadecimal
              notation.

              Permission to access this file is governed by a ptrace access
              mode PTRACE_MODE_ATTACH_FSCREDS check; see ptrace(2).

       /proc/[pid]/root
              UNIX and Linux support the idea of a per-process root of the
              filesystem, set by the chroot(2) system call.  This file is a
              symbolic link that points to the process's root directory, and
              behaves in the same way as exe, and fd/*.

              Note however that this file is not merely a symbolic link.  It
              provides the same view of the filesystem (including namespaces
              and the set of per-process mounts) as the process itself.  An
              example illustrates this point.  In one terminal, we start a
              shell in new user and mount namespaces, and in that shell we
              create some new mount points:

                  $ PS1='sh1# ' unshare -Urnm
                  sh1# mount -t tmpfs tmpfs /etc  # Mount empty tmpfs at /etc
                  sh1# mount --bind /usr /dev     # Mount /usr at /dev
                  sh1# echo $$
                  27123

              In a second terminal window, in the initial mount namespace,
              we look at the contents of the corresponding mounts in the
              initial and new namespaces:

                  $ PS1='sh2# ' sudo sh
                  sh2# ls /etc | wc -l                  # In initial NS
                  309
                  sh2# ls /proc/27123/root/etc | wc -l  # /etc in other NS
                  0                                     # The empty tmpfs dir
                  sh2# ls /dev | wc -l                  # In initial NS
                  205
                  sh2# ls /proc/27123/root/dev | wc -l  # /dev in other NS
                  11                                    # Actually bind
                                                        # mounted to /usr
                  sh2# ls /usr | wc -l                  # /usr in initial NS
                  11

              In a multithreaded process, the contents of the
              /proc/[pid]/root symbolic link are not available if the main
              thread has already terminated (typically by calling
              pthread_exit(3)).

              Permission to dereference or read (readlink(2)) this symbolic
              link is governed by a ptrace access mode
              PTRACE_MODE_READ_FSCREDS check; see ptrace(2).

       /proc/[pid]/seccomp (Linux 2.6.12 to 2.6.22)
              This file can be used to read and change the process's secure
              computing (seccomp) mode setting.  It contains the value 0 if
              the process is not in seccomp mode, and 1 if the process is in
              strict seccomp mode (see seccomp(2)).  Writing 1 to this file
              places the process irreversibly in strict seccomp mode.  (Fur‐
              ther attempts to write to the file fail with the EPERM error.)

              In Linux 2.6.23, this file went away, to be replaced by the
              prctl(2) PR_GET_SECCOMP and PR_SET_SECCOMP operations (and
              later by seccomp(2) and the Seccomp field in /proc/[pid]/sta‐
              tus).

       /proc/[pid]/setgroups (since Linux 3.19)
              See user_namespaces(7).

       /proc/[pid]/smaps (since Linux 2.6.14)
              This file shows memory consumption for each of the process's
              mappings.  (The pmap(1) command displays similar information,
              in a form that may be easier for parsing.)  For each mapping
              there is a series of lines such as the following:

                  00400000-0048a000 r-xp 00000000 fd:03 960637       /bin/bash
                  Size:                552 kB
                  Rss:                 460 kB
                  Pss:                 100 kB
                  Shared_Clean:        452 kB
                  Shared_Dirty:          0 kB
                  Private_Clean:         8 kB
                  Private_Dirty:         0 kB
                  Referenced:          460 kB
                  Anonymous:             0 kB
                  AnonHugePages:         0 kB
                  ShmemHugePages:        0 kB
                  ShmemPmdMapped:        0 kB
                  Swap:                  0 kB
                  KernelPageSize:        4 kB
                  MMUPageSize:           4 kB
                  KernelPageSize:        4 kB
                  MMUPageSize:           4 kB
                  Locked:                0 kB
                  ProtectionKey:         0
                  VmFlags: rd ex mr mw me dw

              The first of these lines shows the same information as is dis‐
              played for the mapping in /proc/[pid]/maps.  The following
              lines show the size of the mapping, the amount of the mapping
              that is currently resident in RAM ("Rss"), the process's pro‐
              portional share of this mapping ("Pss"), the number of clean
              and dirty shared pages in the mapping, and the number of clean
              and dirty private pages in the mapping.  "Referenced" indi‐
              cates the amount of memory currently marked as referenced or
              accessed.  "Anonymous" shows the amount of memory that does
              not belong to any file.  "Swap" shows how much would-be-anony‐
              mous memory is also used, but out on swap.

              The "KernelPageSize" line (available since Linux 2.6.29) is
              the page size used by the kernel to back the virtual memory
              area.  This matches the size used by the MMU in the majority
              of cases.  However, one counter-example occurs on PPC64 ker‐
              nels whereby a kernel using 64kB as a base page size may still
              use 4kB pages for the MMU on older processors.  To distinguish
              the two attributes, the "MMUPageSize" line (also available
              since Linux 2.6.29) reports the page size used by the MMU.

              The "Locked" indicates whether the mapping is locked in memory
              or not.

              The "ProtectionKey" line (available since Linux 4.9, on x86
              only) contains the memory protection key (see pkeys(7)) asso‐
              ciated with the virtual memory area.  This entry is present
              only if the kernel was built with the CONFIG_X86_INTEL_MEM‐
              ORY_PROTECTION_KEYS configuration option.

              The "VmFlags" line (available since Linux 3.8) represents the
              kernel flags associated with the virtual memory area, encoded
              using the following two-letter codes:

                  rd  - readable
                  wr  - writable
                  ex  - executable
                  sh  - shared
                  mr  - may read
                  mw  - may write
                  me  - may execute
                  ms  - may share
                  gd  - stack segment grows down
                  pf  - pure PFN range
                  dw  - disabled write to the mapped file
                  lo  - pages are locked in memory
                  io  - memory mapped I/O area
                  sr  - sequential read advise provided
                  rr  - random read advise provided
                  dc  - do not copy area on fork
                  de  - do not expand area on remapping
                  ac  - area is accountable
                  nr  - swap space is not reserved for the area
                  ht  - area uses huge tlb pages
                  nl  - non-linear mapping
                  ar  - architecture specific flag
                  dd  - do not include area into core dump
                  sd  - soft-dirty flag
                  mm  - mixed map area
                  hg  - huge page advise flag
                  nh  - no-huge page advise flag
                  mg  - mergeable advise flag

              "ProtectionKey" field contains the memory protection key (see
              pkeys(5)) associated with the virtual memory area.  Present
              only if the kernel was built with the CONFIG_X86_INTEL_MEM‐
              ORY_PROTECTION_KEYS configuration option. (since Linux 4.6)

              The /proc/[pid]/smaps file is present only if the CON‐
              FIG_PROC_PAGE_MONITOR kernel configuration option is enabled.

       /proc/[pid]/stack (since Linux 2.6.29)
              This file provides a symbolic trace of the function calls in
              this process's kernel stack.  This file is provided only if
              the kernel was built with the CONFIG_STACKTRACE configuration
              option.

              Permission to access this file is governed by a ptrace access
              mode PTRACE_MODE_ATTACH_FSCREDS check; see ptrace(2).

       /proc/[pid]/stat
              Status information about the process.  This is used by ps(1).
              It is defined in the kernel source file fs/proc/array.c.

              The fields, in order, with their proper scanf(3) format speci‐
              fiers, are listed below.  Whether or not certain of these
              fields display valid information is governed by a ptrace
              access mode PTRACE_MODE_READ_FSCREDS | PTRACE_MODE_NOAUDIT
              check (refer to ptrace(2)).  If the check denies access, then
              the field value is displayed as 0.  The affected fields are
              indicated with the marking [PT].

              (1) pid  %d
                        The process ID.

              (2) comm  %s
                        The filename of the executable, in parentheses.
                        This is visible whether or not the executable is
                        swapped out.

              (3) state  %c
                        One of the following characters, indicating process
                        state:

                        R  Running

                        S  Sleeping in an interruptible wait

                        D  Waiting in uninterruptible disk sleep

                        Z  Zombie

                        T  Stopped (on a signal) or (before Linux 2.6.33)
                           trace stopped

                        t  Tracing stop (Linux 2.6.33 onward)

                        W  Paging (only before Linux 2.6.0)

                        X  Dead (from Linux 2.6.0 onward)

                        x  Dead (Linux 2.6.33 to 3.13 only)

                        K  Wakekill (Linux 2.6.33 to 3.13 only)

                        W  Waking (Linux 2.6.33 to 3.13 only)

                        P  Parked (Linux 3.9 to 3.13 only)

              (4) ppid  %d
                        The PID of the parent of this process.

              (5) pgrp  %d
                        The process group ID of the process.

              (6) session  %d
                        The session ID of the process.

              (7) tty_nr  %d
                        The controlling terminal of the process.  (The minor
                        device number is contained in the combination of
                        bits 31 to 20 and 7 to 0; the major device number is
                        in bits 15 to 8.)

              (8) tpgid  %d
                        The ID of the foreground process group of the con‐
                        trolling terminal of the process.

              (9) flags  %u
                        The kernel flags word of the process.  For bit mean‐
                        ings, see the PF_* defines in the Linux kernel
                        source file include/linux/sched.h.  Details depend
                        on the kernel version.

                        The format for this field was %lu before Linux 2.6.

              (10) minflt  %lu
                        The number of minor faults the process has made
                        which have not required loading a memory page from
                        disk.

              (11) cminflt  %lu
                        The number of minor faults that the process's
                        waited-for children have made.

              (12) majflt  %lu
                        The number of major faults the process has made
                        which have required loading a memory page from disk.

              (13) cmajflt  %lu
                        The number of major faults that the process's
                        waited-for children have made.

              (14) utime  %lu
                        Amount of time that this process has been scheduled
                        in user mode, measured in clock ticks (divide by
                        sysconf(_SC_CLK_TCK)).  This includes guest time,
                        guest_time (time spent running a virtual CPU, see
                        below), so that applications that are not aware of
                        the guest time field do not lose that time from
                        their calculations.

              (15) stime  %lu
                        Amount of time that this process has been scheduled
                        in kernel mode, measured in clock ticks (divide by
                        sysconf(_SC_CLK_TCK)).

              (16) cutime  %ld
                        Amount of time that this process's waited-for chil‐
                        dren have been scheduled in user mode, measured in
                        clock ticks (divide by sysconf(_SC_CLK_TCK)).  (See
                        also times(2).)  This includes guest time,
                        cguest_time (time spent running a virtual CPU, see
                        below).

              (17) cstime  %ld
                        Amount of time that this process's waited-for chil‐
                        dren have been scheduled in kernel mode, measured in
                        clock ticks (divide by sysconf(_SC_CLK_TCK)).

              (18) priority  %ld
                        (Explanation for Linux 2.6) For processes running a
                        real-time scheduling policy (policy below; see
                        sched_setscheduler(2)), this is the negated schedul‐
                        ing priority, minus one; that is, a number in the
                        range -2 to -100, corresponding to real-time priori‐
                        ties 1 to 99.  For processes running under a non-
                        real-time scheduling policy, this is the raw nice
                        value (setpriority(2)) as represented in the kernel.
                        The kernel stores nice values as numbers in the
                        range 0 (high) to 39 (low), corresponding to the
                        user-visible nice range of -20 to 19.

                        Before Linux 2.6, this was a scaled value based on
                        the scheduler weighting given to this process.

              (19) nice  %ld
                        The nice value (see setpriority(2)), a value in the
                        range 19 (low priority) to -20 (high priority).

              (20) num_threads  %ld
                        Number of threads in this process (since Linux 2.6).
                        Before kernel 2.6, this field was hard coded to 0 as
                        a placeholder for an earlier removed field.

              (21) itrealvalue  %ld
                        The time in jiffies before the next SIGALRM is sent
                        to the process due to an interval timer.  Since ker‐
                        nel 2.6.17, this field is no longer maintained, and
                        is hard coded as 0.

              (22) starttime  %llu
                        The time the process started after system boot.  In
                        kernels before Linux 2.6, this value was expressed
                        in jiffies.  Since Linux 2.6, the value is expressed
                        in clock ticks (divide by sysconf(_SC_CLK_TCK)).

                        The format for this field was %lu before Linux 2.6.

              (23) vsize  %lu
                        Virtual memory size in bytes.

              (24) rss  %ld
                        Resident Set Size: number of pages the process has
                        in real memory.  This is just the pages which count
                        toward text, data, or stack space.  This does not
                        include pages which have not been demand-loaded in,
                        or which are swapped out.

              (25) rsslim  %lu
                        Current soft limit in bytes on the rss of the
                        process; see the description of RLIMIT_RSS in
                        getrlimit(2).

              (26) startcode  %lu  [PT]
                        The address above which program text can run.

              (27) endcode  %lu  [PT]
                        The address below which program text can run.

              (28) startstack  %lu  [PT]
                        The address of the start (i.e., bottom) of the
                        stack.

              (29) kstkesp  %lu  [PT]
                        The current value of ESP (stack pointer), as found
                        in the kernel stack page for the process.

              (30) kstkeip  %lu  [PT]
                        The current EIP (instruction pointer).

              (31) signal  %lu
                        The bitmap of pending signals, displayed as a deci‐
                        mal number.  Obsolete, because it does not provide
                        information on real-time signals; use
                        /proc/[pid]/status instead.

              (32) blocked  %lu
                        The bitmap of blocked signals, displayed as a deci‐
                        mal number.  Obsolete, because it does not provide
                        information on real-time signals; use
                        /proc/[pid]/status instead.

              (33) sigignore  %lu
                        The bitmap of ignored signals, displayed as a deci‐
                        mal number.  Obsolete, because it does not provide
                        information on real-time signals; use
                        /proc/[pid]/status instead.

              (34) sigcatch  %lu
                        The bitmap of caught signals, displayed as a decimal
                        number.  Obsolete, because it does not provide
                        information on real-time signals; use
                        /proc/[pid]/status instead.

              (35) wchan  %lu  [PT]
                        This is the "channel" in which the process is wait‐
                        ing.  It is the address of a location in the kernel
                        where the process is sleeping.  The corresponding
                        symbolic name can be found in /proc/[pid]/wchan.

              (36) nswap  %lu
                        Number of pages swapped (not maintained).

              (37) cnswap  %lu
                        Cumulative nswap for child processes (not main‐
                        tained).

              (38) exit_signal  %d  (since Linux 2.1.22)
                        Signal to be sent to parent when we die.

              (39) processor  %d  (since Linux 2.2.8)
                        CPU number last executed on.

              (40) rt_priority  %u  (since Linux 2.5.19)
                        Real-time scheduling priority, a number in the range
                        1 to 99 for processes scheduled under a real-time
                        policy, or 0, for non-real-time processes (see
                        sched_setscheduler(2)).

              (41) policy  %u  (since Linux 2.5.19)
                        Scheduling policy (see sched_setscheduler(2)).
                        Decode using the SCHED_* constants in linux/sched.h.

                        The format for this field was %lu before Linux
                        2.6.22.

              (42) delayacct_blkio_ticks  %llu  (since Linux 2.6.18)
                        Aggregated block I/O delays, measured in clock ticks
                        (centiseconds).

              (43) guest_time  %lu  (since Linux 2.6.24)
                        Guest time of the process (time spent running a vir‐
                        tual CPU for a guest operating system), measured in
                        clock ticks (divide by sysconf(_SC_CLK_TCK)).

              (44) cguest_time  %ld  (since Linux 2.6.24)
                        Guest time of the process's children, measured in
                        clock ticks (divide by sysconf(_SC_CLK_TCK)).

              (45) start_data  %lu  (since Linux 3.3)  [PT]
                        Address above which program initialized and unini‐
                        tialized (BSS) data are placed.

              (46) end_data  %lu  (since Linux 3.3)  [PT]
                        Address below which program initialized and unini‐
                        tialized (BSS) data are placed.

              (47) start_brk  %lu  (since Linux 3.3)  [PT]
                        Address above which program heap can be expanded
                        with brk(2).

              (48) arg_start  %lu  (since Linux 3.5)  [PT]
                        Address above which program command-line arguments
                        (argv) are placed.

              (49) arg_end  %lu  (since Linux 3.5)  [PT]
                        Address below program command-line arguments (argv)
                        are placed.

              (50) env_start  %lu  (since Linux 3.5)  [PT]
                        Address above which program environment is placed.

              (51) env_end  %lu  (since Linux 3.5)  [PT]
                        Address below which program environment is placed.

              (52) exit_code  %d  (since Linux 3.5)  [PT]
                        The thread's exit status in the form reported by
                        waitpid(2).

       /proc/[pid]/statm
              Provides information about memory usage, measured in pages.
              The columns are:

                  size       (1) total program size
                             (same as VmSize in /proc/[pid]/status)
                  resident   (2) resident set size
                             (same as VmRSS in /proc/[pid]/status)
                  shared     (3) number of resident shared pages (i.e., backed by a file)
                             (same as RssFile+RssShmem in /proc/[pid]/status)
                  text       (4) text (code)
                  lib        (5) library (unused since Linux 2.6; always 0)
                  data       (6) data + stack
                  dt         (7) dirty pages (unused since Linux 2.6; always 0)

       /proc/[pid]/status
              Provides much of the information in /proc/[pid]/stat and
              /proc/[pid]/statm in a format that's easier for humans to
              parse.  Here's an example:

                  $ cat /proc/$$/status
                  Name:   bash
                  Umask:  0022
                  State:  S (sleeping)
                  Tgid:   17248
                  Ngid:   0
                  Pid:    17248
                  PPid:   17200
                  TracerPid:      0
                  Uid:    1000    1000    1000    1000
                  Gid:    100     100     100     100
                  FDSize: 256
                  Groups: 16 33 100
                  NStgid: 17248
                  NSpid:  17248
                  NSpgid: 17248
                  NSsid:  17200
                  VmPeak:     131168 kB
                  VmSize:     131168 kB
                  VmLck:           0 kB
                  VmPin:           0 kB
                  VmHWM:       13484 kB
                  VmRSS:       13484 kB
                  RssAnon:     10264 kB
                  RssFile:      3220 kB
                  RssShmem:        0 kB
                  VmData:      10332 kB
                  VmStk:         136 kB
                  VmExe:         992 kB
                  VmLib:        2104 kB
                  VmPTE:          76 kB
                  VmPMD:          12 kB
                  VmSwap:          0 kB
                  HugetlbPages:          0 kB        # 4.4
                  Threads:        1
                  SigQ:   0/3067
                  SigPnd: 0000000000000000
                  ShdPnd: 0000000000000000
                  SigBlk: 0000000000010000
                  SigIgn: 0000000000384004
                  SigCgt: 000000004b813efb
                  CapInh: 0000000000000000
                  CapPrm: 0000000000000000
                  CapEff: 0000000000000000
                  CapBnd: ffffffffffffffff
                  CapAmb:   0000000000000000
                  NoNewPrivs:     0
                  Seccomp:        0
                  Cpus_allowed:   00000001
                  Cpus_allowed_list:      0
                  Mems_allowed:   1
                  Mems_allowed_list:      0
                  voluntary_ctxt_switches:        150
                  nonvoluntary_ctxt_switches:     545

              The fields are as follows:

              * Name: Command run by this process.

              * Umask: Process umask, expressed in octal with a leading
                zero; see umask(2).  (Since Linux 4.7.)

              * State: Current state of the process.  One of "R (running)",
                "S (sleeping)", "D (disk sleep)", "T (stopped)", "T (tracing
                stop)", "Z (zombie)", or "X (dead)".

              * Tgid: Thread group ID (i.e., Process ID).

              * Ngid: NUMA group ID (0 if none; since Linux 3.13).

              * Pid: Thread ID (see gettid(2)).

              * PPid: PID of parent process.

              * TracerPid: PID of process tracing this process (0 if not
                being traced).

              * Uid, Gid: Real, effective, saved set, and filesystem UIDs
                (GIDs).

              * FDSize: Number of file descriptor slots currently allocated.

              * Groups: Supplementary group list.

              * NStgid : Thread group ID (i.e., PID) in each of the PID
                namespaces of which [pid] is a member.  The leftmost entry
                shows the value with respect to the PID namespace of the
                reading process, followed by the value in successively
                nested inner namespaces.  (Since Linux 4.1.)

              * NSpid: Thread ID in each of the PID namespaces of which
                [pid] is a member.  The fields are ordered as for NStgid.
                (Since Linux 4.1.)

              * NSpgid: Process group ID in each of the PID namespaces of
                which [pid] is a member.  The fields are ordered as for NSt‐
                gid.  (Since Linux 4.1.)

              * NSsid: descendant namespace session ID hierarchy Session ID
                in each of the PID namespaces of which [pid] is a member.
                The fields are ordered as for NStgid.  (Since Linux 4.1.)

              * VmPeak: Peak virtual memory size.

              * VmSize: Virtual memory size.

              * VmLck: Locked memory size (see mlock(3)).

              * VmPin: Pinned memory size (since Linux 3.2).  These are
                pages that can't be moved because something needs to
                directly access physical memory.

              * VmHWM: Peak resident set size ("high water mark").

              * VmRSS: Resident set size.  Note that the value here is the
                sum of RssAnon, RssFile, and RssShmem.

              * RssAnon: Size of resident anonymous memory.  (since Linux
                4.5).

              * RssFile: Size of resident file mappings.  (since Linux 4.5).

              * RssShmem: Size of resident shared memory (includes System V
                shared memory, mappings from tmpfs(5), and shared anonymous
                mappings).  (since Linux 4.5).

              * VmData, VmStk, VmExe: Size of data, stack, and text seg‐
                ments.

              * VmLib: Shared library code size.

              * VmPTE: Page table entries size (since Linux 2.6.10).

              * VmPMD: Size of second-level page tables (since Linux 4.0).

              * VmSwap: Swapped-out virtual memory size by anonymous private
                pages; shmem swap usage is not included (since Linux
                2.6.34).

              * HugetlbPages: Size of hugetlb memory portions.  (since Linux
                4.4).

              * Threads: Number of threads in process containing this
                thread.

              * SigQ: This field contains two slash-separated numbers that
                relate to queued signals for the real user ID of this
                process.  The first of these is the number of currently
                queued signals for this real user ID, and the second is the
                resource limit on the number of queued signals for this
                process (see the description of RLIMIT_SIGPENDING in
                getrlimit(2)).

              * SigPnd, ShdPnd: Number of signals pending for thread and for
                process as a whole (see pthreads(7) and signal(7)).

              * SigBlk, SigIgn, SigCgt: Masks indicating signals being
                blocked, ignored, and caught (see signal(7)).

              * CapInh, CapPrm, CapEff: Masks of capabilities enabled in
                inheritable, permitted, and effective sets (see
                capabilities(7)).

              * CapBnd: Capability Bounding set (since Linux 2.6.26, see
                capabilities(7)).

              * CapAmb: Ambient capability set (since Linux 4.3, see
                capabilities(7)).

              * NoNewPrivs: Value of the no_new_privs bit (since Linux 4.10,
                see prctl(2)).

              * Seccomp: Seccomp mode of the process (since Linux 3.8, see
                seccomp(2)).  0 means SECCOMP_MODE_DISABLED; 1 means SEC‐
                COMP_MODE_STRICT; 2 means SECCOMP_MODE_FILTER.  This field
                is provided only if the kernel was built with the CON‐
                FIG_SECCOMP kernel configuration option enabled.

              * Cpus_allowed: Mask of CPUs on which this process may run
                (since Linux 2.6.24, see cpuset(7)).

              * Cpus_allowed_list: Same as previous, but in "list format"
                (since Linux 2.6.26, see cpuset(7)).

              * Mems_allowed: Mask of memory nodes allowed to this process
                (since Linux 2.6.24, see cpuset(7)).

              * Mems_allowed_list: Same as previous, but in "list format"
                (since Linux 2.6.26, see cpuset(7)).

              * voluntary_ctxt_switches, nonvoluntary_ctxt_switches: Number
                of voluntary and involuntary context switches (since Linux
                2.6.23).

       /proc/[pid]/syscall (since Linux 2.6.27)
              This file exposes the system call number and argument regis‐
              ters for the system call currently being executed by the
              process, followed by the values of the stack pointer and pro‐
              gram counter registers.  The values of all six argument regis‐
              ters are exposed, although most system calls use fewer regis‐
              ters.

              If the process is blocked, but not in a system call, then the
              file displays -1 in place of the system call number, followed
              by just the values of the stack pointer and program counter.
              If process is not blocked, then the file contains just the
              string "running".

              This file is present only if the kernel was configured with
              CONFIG_HAVE_ARCH_TRACEHOOK.

              Permission to access this file is governed by a ptrace access
              mode PTRACE_MODE_ATTACH_FSCREDS check; see ptrace(2).

       /proc/[pid]/task (since Linux 2.6.0-test6)
              This is a directory that contains one subdirectory for each
              thread in the process.  The name of each subdirectory is the
              numerical thread ID ([tid]) of the thread (see gettid(2)).
              Within each of these subdirectories, there is a set of files
              with the same names and contents as under the /proc/[pid]
              directories.  For attributes that are shared by all threads,
              the contents for each of the files under the task/[tid] subdi‐
              rectories will be the same as in the corresponding file in the
              parent /proc/[pid] directory (e.g., in a multithreaded
              process, all of the task/[tid]/cwd files will have the same
              value as the /proc/[pid]/cwd file in the parent directory,
              since all of the threads in a process share a working direc‐
              tory).  For attributes that are distinct for each thread, the
              corresponding files under task/[tid] may have different values
              (e.g., various fields in each of the task/[tid]/status files
              may be different for each thread), or they might not exist in
              /proc/[pid] at all.  In a multithreaded process, the contents
              of the /proc/[pid]/task directory are not available if the
              main thread has already terminated (typically by calling
              pthread_exit(3)).

       /proc/[pid]/task/[tid]/children (since Linux 3.5)
              A space-separated list of child tasks of this task.  Each
              child task is represented by its TID.

              This option is intended for use by the checkpoint-restore
              (CRIU) system, and reliably provides a list of children only
              if all of the child processes are stopped or frozen.  It does
              not work properly if children of the target task exit while
              the file is being read!  Exiting children may cause non-exit‐
              ing children to be omitted from the list.  This makes this
              interface even more unreliable than classic PID-based
              approaches if the inspected task and its children aren't
              frozen, and most code should probably not use this interface.

              Until Linux 4.2, the presence of this file was governed by the
              CONFIG_CHECKPOINT_RESTORE kernel configuration option.  Since
              Linux 4.2, it is governed by the CONFIG_PROC_CHILDREN option.

       /proc/[pid]/timers (since Linux 3.10)
              A list of the POSIX timers for this process.  Each timer is
              listed with a line that starts with the string "ID:".  For
              example:

                  ID: 1
                  signal: 60/00007fff86e452a8
                  notify: signal/pid.2634
                  ClockID: 0
                  ID: 0
                  signal: 60/00007fff86e452a8
                  notify: signal/pid.2634
                  ClockID: 1

              The lines shown for each timer have the following meanings:

              ID     The ID for this timer.  This is not the same as the
                     timer ID returned by timer_create(2); rather, it is the
                     same kernel-internal ID that is available via the
                     si_timerid field of the siginfo_t structure (see
                     sigaction(2)).

              signal This is the signal number that this timer uses to
                     deliver notifications followed by a slash, and then the
                     sigev_value value supplied to the signal handler.
                     Valid only for timers that notify via a signal.

              notify The part before the slash specifies the mechanism that
                     this timer uses to deliver notifications, and is one of
                     "thread", "signal", or "none".  Immediately following
                     the slash is either the string "tid" for timers with
                     SIGEV_THREAD_ID notification, or "pid" for timers that
                     notify by other mechanisms.  Following the "." is the
                     PID of the process (or the kernel thread ID of the
                     thread)  that will be delivered a signal if the timer
                     delivers notifications via a signal.

              ClockID
                     This field identifies the clock that the timer uses for
                     measuring time.  For most clocks, this is a number that
                     matches one of the user-space CLOCK_* constants exposed
                     via <time.h>.  CLOCK_PROCESS_CPUTIME_ID timers display
                     with a value of -6 in this field.
                     CLOCK_THREAD_CPUTIME_ID timers display with a value of
                     -2 in this field.

              This file is available only when the kernel was configured
              with CONFIG_CHECKPOINT_RESTORE.

       /proc/[pid]/timerslack_ns (since Linux 4.6)
              This file exposes the process's "current" timer slack value,
              expressed in nanoseconds.  The file is writable, allowing the
              process's timer slack value to be changed.  Writing 0 to this
              file resets the "current" timer slack to the "default" timer
              slack value.  For further details, see the discussion of
              PR_SET_TIMERSLACK in prctl(2).

              Initially, permission to access this file was governed by a
              ptrace access mode PTRACE_MODE_ATTACH_FSCREDS check (see
              ptrace(2)).  However, this was subsequently deemed too strict
              a requirement (and had the side effect that requiring a
              process to have the CAP_SYS_PTRACE capability would also allow
              it to view and change any process's memory).  Therefore, since
              Linux 4.9, only the (weaker) CAP_SYS_NICE capability is
              required to access this file.

       /proc/[pid]/uid_map, /proc/[pid]/gid_map (since Linux 3.5)
              See user_namespaces(7).

       /proc/[pid]/wchan (since Linux 2.6.0)
              The symbolic name corresponding to the location in the kernel
              where the process is sleeping.

              Permission to access this file is governed by a ptrace access
              mode PTRACE_MODE_READ_FSCREDS check; see ptrace(2).

       /proc/apm
              Advanced power management version and battery information when
              CONFIG_APM is defined at kernel compilation time.

       /proc/buddyinfo
              This file contains information which is used for diagnosing
              memory fragmentation issues.  Each line starts with the iden‐
              tification of the node and the name of the zone which together
              identify a memory region This is then followed by the count of
              available chunks of a certain order in which these zones are
              split.  The size in bytes of a certain order is given by the
              formula:

                  (2^order) * PAGE_SIZE

              The binary buddy allocator algorithm inside the kernel will
              split one chunk into two chunks of a smaller order (thus with
              half the size) or combine two contiguous chunks into one
              larger chunk of a higher order (thus with double the size) to
              satisfy allocation requests and to counter memory fragmenta‐
              tion.  The order matches the column number, when starting to
              count at zero.

              For example on an x86-64 system:

  Node 0, zone     DMA     1    1    1    0    2    1    1    0    1    1    3
  Node 0, zone   DMA32    65   47    4   81   52   28   13   10    5    1  404
  Node 0, zone  Normal   216   55  189  101   84   38   37   27    5    3  587

              In this example, there is one node containing three zones and
              there are 11 different chunk sizes.  If the page size is 4
              kilobytes, then the first zone called DMA (on x86 the first 16
              megabyte of memory) has 1 chunk of 4 kilobytes (order 0)
              available and has 3 chunks of 4 megabytes (order 10) avail‐
              able.

              If the memory is heavily fragmented, the counters for higher
              order chunks will be zero and allocation of large contiguous
              areas will fail.

              Further information about the zones can be found in
              /proc/zoneinfo.

       /proc/bus
              Contains subdirectories for installed busses.

       /proc/bus/pccard
              Subdirectory for PCMCIA devices when CONFIG_PCMCIA is set at
              kernel compilation time.

       /proc/bus/pccard/drivers

       /proc/bus/pci
              Contains various bus subdirectories and pseudo-files contain‐
              ing information about PCI busses, installed devices, and
              device drivers.  Some of these files are not ASCII.

       /proc/bus/pci/devices
              Information about PCI devices.  They may be accessed through
              lspci(8) and setpci(8).

       /proc/cgroups (since Linux 2.6.24)
              See cgroups(7).

       /proc/cmdline
              Arguments passed to the Linux kernel at boot time.  Often done
              via a boot manager such as lilo(8) or grub(8).

       /proc/config.gz (since Linux 2.6)
              This file exposes the configuration options that were used to
              build the currently running kernel, in the same format as they
              would be shown in the .config file that resulted when config‐
              uring the kernel (using make xconfig, make config, or simi‐
              lar).  The file contents are compressed; view or search them
              using zcat(1) and zgrep(1).  As long as no changes have been
              made to the following file, the contents of /proc/config.gz
              are the same as those provided by:

                  cat /lib/modules/$(uname -r)/build/.config

              /proc/config.gz is provided only if the kernel is configured
              with CONFIG_IKCONFIG_PROC.

       /proc/crypto
              A list of the ciphers provided by the kernel crypto API.  For
              details, see the kernel Linux Kernel Crypto API documentation
              available under the kernel source directory Documenta‐
              tion/crypto/ (or Documentation/DocBook before 4.10; the docu‐
              mentation can be built using a command such as make htmldocs
              in the root directory of the kernel source tree).

       /proc/cpuinfo
              This is a collection of CPU and system architecture dependent
              items, for each supported architecture a different list.  Two
              common entries are processor which gives CPU number and
              bogomips; a system constant that is calculated during kernel
              initialization.  SMP machines have information for each CPU.
              The lscpu(1) command gathers its information from this file.

       /proc/devices
              Text listing of major numbers and device groups.  This can be
              used by MAKEDEV scripts for consistency with the kernel.

       /proc/diskstats (since Linux 2.5.69)
              This file contains disk I/O statistics for each disk device.
              See the Linux kernel source file Documentation/iostats.txt for
              further information.

       /proc/dma
              This is a list of the registered ISA DMA (direct memory
              access) channels in use.

       /proc/driver
              Empty subdirectory.

       /proc/execdomains
              List of the execution domains (ABI personalities).

       /proc/fb
              Frame buffer information when CONFIG_FB is defined during ker‐
              nel compilation.

       /proc/filesystems
              A text listing of the filesystems which are supported by the
              kernel, namely filesystems which were compiled into the kernel
              or whose kernel modules are currently loaded.  (See also
              filesystems(5).)  If a filesystem is marked with "nodev", this
              means that it does not require a block device to be mounted
              (e.g., virtual filesystem, network filesystem).

              Incidentally, this file may be used by mount(8) when no
              filesystem is specified and it didn't manage to determine the
              filesystem type.  Then filesystems contained in this file are
              tried (excepted those that are marked with "nodev").

       /proc/fs
              Contains subdirectories that in turn contain files with infor‐
              mation about (certain) mounted filesystems.

       /proc/ide
              This directory exists on systems with the IDE bus.  There are
              directories for each IDE channel and attached device.  Files
              include:

                  cache              buffer size in KB
                  capacity           number of sectors
                  driver             driver version
                  geometry           physical and logical geometry
                  identify           in hexadecimal
                  media              media type
                  model              manufacturer's model number
                  settings           drive settings
                  smart_thresholds   in hexadecimal
                  smart_values       in hexadecimal

              The hdparm(8) utility provides access to this information in a
              friendly format.

       /proc/interrupts
              This is used to record the number of interrupts per CPU per IO
              device.  Since Linux 2.6.24, for the i386 and x86-64 architec‐
              tures, at least, this also includes interrupts internal to the
              system (that is, not associated with a device as such), such
              as NMI (nonmaskable interrupt), LOC (local timer interrupt),
              and for SMP systems, TLB (TLB flush interrupt), RES
              (rescheduling interrupt), CAL (remote function call inter‐
              rupt), and possibly others.  Very easy to read formatting,
              done in ASCII.

       /proc/iomem
              I/O memory map in Linux 2.4.

       /proc/ioports
              This is a list of currently registered Input-Output port
              regions that are in use.

       /proc/kallsyms (since Linux 2.5.71)
              This holds the kernel exported symbol definitions used by the
              modules(X) tools to dynamically link and bind loadable mod‐
              ules.  In Linux 2.5.47 and earlier, a similar file with
              slightly different syntax was named ksyms.

       /proc/kcore
              This file represents the physical memory of the system and is
              stored in the ELF core file format.  With this pseudo-file,
              and an unstripped kernel (/usr/src/linux/vmlinux) binary, GDB
              can be used to examine the current state of any kernel data
              structures.

              The total length of the file is the size of physical memory
              (RAM) plus 4 KiB.

       /proc/keys (since Linux 2.6.10)
              See keyrings(7).

       /proc/key-users (since Linux 2.6.10)
              See keyrings(7).

       /proc/kmsg
              This file can be used instead of the syslog(2) system call to
              read kernel messages.  A process must have superuser privi‐
              leges to read this file, and only one process should read this
              file.  This file should not be read if a syslog process is
              running which uses the syslog(2) system call facility to log
              kernel messages.

              Information in this file is retrieved with the dmesg(1) pro‐
              gram.

       /proc/kpagecgroup (since Linux 4.3)
              This file contains a 64-bit inode number of the memory cgroup
              each page is charged to, indexed by page frame number (see the
              discussion of /proc/[pid]/pagemap).

              The /proc/kpagecgroup file is present only if the CONFIG_MEMCG
              kernel configuration option is enabled.

       /proc/kpagecount (since Linux 2.6.25)
              This file contains a 64-bit count of the number of times each
              physical page frame is mapped, indexed by page frame number
              (see the discussion of /proc/[pid]/pagemap).

              The /proc/kpagecount file is present only if the CON‐
              FIG_PROC_PAGE_MONITOR kernel configuration option is enabled.

       /proc/kpageflags (since Linux 2.6.25)
              This file contains 64-bit masks corresponding to each physical
              page frame; it is indexed by page frame number (see the dis‐
              cussion of /proc/[pid]/pagemap).  The bits are as follows:

                   0 - KPF_LOCKED
                   1 - KPF_ERROR
                   2 - KPF_REFERENCED
                   3 - KPF_UPTODATE
                   4 - KPF_DIRTY
                   5 - KPF_LRU
                   6 - KPF_ACTIVE
                   7 - KPF_SLAB
                   8 - KPF_WRITEBACK
                   9 - KPF_RECLAIM
                  10 - KPF_BUDDY
                  11 - KPF_MMAP           (since Linux 2.6.31)
                  12 - KPF_ANON           (since Linux 2.6.31)
                  13 - KPF_SWAPCACHE      (since Linux 2.6.31)
                  14 - KPF_SWAPBACKED     (since Linux 2.6.31)
                  15 - KPF_COMPOUND_HEAD  (since Linux 2.6.31)
                  16 - KPF_COMPOUND_TAIL  (since Linux 2.6.31)
                  17 - KPF_HUGE           (since Linux 2.6.31)
                  18 - KPF_UNEVICTABLE    (since Linux 2.6.31)
                  19 - KPF_HWPOISON       (since Linux 2.6.31)
                  20 - KPF_NOPAGE         (since Linux 2.6.31)
                  21 - KPF_KSM            (since Linux 2.6.32)
                  22 - KPF_THP            (since Linux 3.4)
                  23 - KPF_BALLOON        (since Linux 3.18)
                  24 - KPF_ZERO_PAGE      (since Linux 4.0)
                  25 - KPF_IDLE           (since Linux 4.3)

              For further details on the meanings of these bits, see the
              kernel source file Documentation/vm/pagemap.txt.  Before ker‐
              nel 2.6.29, KPF_WRITEBACK, KPF_RECLAIM, KPF_BUDDY, and
              KPF_LOCKED did not report correctly.

              The /proc/kpageflags file is present only if the CON‐
              FIG_PROC_PAGE_MONITOR kernel configuration option is enabled.

       /proc/ksyms (Linux 1.1.23–2.5.47)
              See /proc/kallsyms.

       /proc/loadavg
              The first three fields in this file are load average figures
              giving the number of jobs in the run queue (state R) or wait‐
              ing for disk I/O (state D) averaged over 1, 5, and 15 minutes.
              They are the same as the load average numbers given by
              uptime(1) and other programs.  The fourth field consists of
              two numbers separated by a slash (/).  The first of these is
              the number of currently runnable kernel scheduling entities
              (processes, threads).  The value after the slash is the number
              of kernel scheduling entities that currently exist on the sys‐
              tem.  The fifth field is the PID of the process that was most
              recently created on the system.

       /proc/locks
              This file shows current file locks (flock(2) and fcntl(2)) and
              leases (fcntl(2)).

              An example of the content shown in this file is the following:

                  1: POSIX  ADVISORY  READ  5433 08:01:7864448 128 128
                  2: FLOCK  ADVISORY  WRITE 2001 08:01:7864554 0 EOF
                  3: FLOCK  ADVISORY  WRITE 1568 00:2f:32388 0 EOF
                  4: POSIX  ADVISORY  WRITE 699 00:16:28457 0 EOF
                  5: POSIX  ADVISORY  WRITE 764 00:16:21448 0 0
                  6: POSIX  ADVISORY  READ  3548 08:01:7867240 1 1
                  7: POSIX  ADVISORY  READ  3548 08:01:7865567 1826 2335
                  8: OFDLCK ADVISORY  WRITE -1 08:01:8713209 128 191

              The fields shown in each line are as follows:

              (1) The ordinal position of the lock in the list.

              (2) The lock type.  Values that may appear here include:

                  FLOCK  This is a BSD file lock created using flock(2).

                  OFDLCK This is an open file description (OFD) lock created
                         using fcntl(2).

                  POSIX  This is a POSIX byte-range lock created using
                         fcntl(2).

              (3) Among the strings that can appear here are the following:

                  ADVISORY
                         This is an advisory lock.

                  MANDATORY
                         This is a mandatory lock.

              (4) The type of lock.  Values that can appear here are:

                  READ   This is a POSIX or OFD read lock, or a BSD shared
                         lock.

                  WRITE  This is a POSIX or OFD write lock, or a BSD exclu‐
                         sive lock.

              (5) The PID of the process that owns the lock.

                  Because OFD locks are not owned by a single process (since
                  multiple processes may have file descriptors that refer to
                  the same open file description), the value -1 is displayed
                  in this field for OFD locks.  (Before kernel 4.14, a bug
                  meant that the PID of the process that initially acquired
                  the lock was displayed instead of the value -1.)

              (6) Three colon-separated subfields that identify the major
                  and minor device ID of the device containing the filesys‐
                  tem where the locked file resides, followed by the inode
                  number of the locked file.

              (7) The byte offset of the first byte of the lock.  For BSD
                  locks, this value is always 0.

              (8) The byte offset of the last byte of the lock.  EOF in this
                  field means that the lock extends to the end of the file.
                  For BSD locks, the value shown is always EOF.

              Since Linux 4.9, the list of locks shown in /proc/locks is
              filtered to show just the locks for the processes in the PID
              namespace (see pid_namespaces(7)) for which the /proc filesys‐
              tem was mounted.  (In the initial PID namespace, there is no
              filtering of the records shown in this file.)

              The lslocks(8) command provides a bit more information about
              each lock.

       /proc/malloc (only up to and including Linux 2.2)
              This file is present only if CONFIG_DEBUG_MALLOC was defined
              during compilation.

       /proc/meminfo
              This file reports statistics about memory usage on the system.
              It is used by free(1) to report the amount of free and used
              memory (both physical and swap) on the system as well as the
              shared memory and buffers used by the kernel.  Each line of
              the file consists of a parameter name, followed by a colon,
              the value of the parameter, and an option unit of measurement
              (e.g., "kB").  The list below describes the parameter names
              and the format specifier required to read the field value.
              Except as noted below, all of the fields have been present
              since at least Linux 2.6.0.  Some fields are displayed only if
              the kernel was configured with various options; those depen‐
              dencies are noted in the list.

              MemTotal %lu
                     Total usable RAM (i.e., physical RAM minus a few
                     reserved bits and the kernel binary code).

              MemFree %lu
                     The sum of LowFree+HighFree.

              MemAvailable %lu (since Linux 3.14)
                     An estimate of how much memory is available for start‐
                     ing new applications, without swapping.

              Buffers %lu
                     Relatively temporary storage for raw disk blocks that
                     shouldn't get tremendously large (20MB or so).

              Cached %lu
                     In-memory cache for files read from the disk (the page
                     cache).  Doesn't include SwapCached.

              SwapCached %lu
                     Memory that once was swapped out, is swapped back in
                     but still also is in the swap file.  (If memory pres‐
                     sure is high, these pages don't need to be swapped out
                     again because they are already in the swap file.  This
                     saves I/O.)

              Active %lu
                     Memory that has been used more recently and usually not
                     reclaimed unless absolutely necessary.

              Inactive %lu
                     Memory which has been less recently used.  It is more
                     eligible to be reclaimed for other purposes.

              Active(anon) %lu (since Linux 2.6.28)
                     [To be documented.]

              Inactive(anon) %lu (since Linux 2.6.28)
                     [To be documented.]

              Active(file) %lu (since Linux 2.6.28)
                     [To be documented.]

              Inactive(file) %lu (since Linux 2.6.28)
                     [To be documented.]

              Unevictable %lu (since Linux 2.6.28)
                     (From Linux 2.6.28 to 2.6.30, CONFIG_UNEVICTABLE_LRU
                     was required.)  [To be documented.]

              Mlocked %lu (since Linux 2.6.28)
                     (From Linux 2.6.28 to 2.6.30, CONFIG_UNEVICTABLE_LRU
                     was required.)  [To be documented.]

              HighTotal %lu
                     (Starting with Linux 2.6.19, CONFIG_HIGHMEM is
                     required.)  Total amount of highmem.  Highmem is all
                     memory above ~860MB of physical memory.  Highmem areas
                     are for use by user-space programs, or for the page
                     cache.  The kernel must use tricks to access this mem‐
                     ory, making it slower to access than lowmem.

              HighFree %lu
                     (Starting with Linux 2.6.19, CONFIG_HIGHMEM is
                     required.)  Amount of free highmem.

              LowTotal %lu
                     (Starting with Linux 2.6.19, CONFIG_HIGHMEM is
                     required.)  Total amount of lowmem.  Lowmem is memory
                     which can be used for everything that highmem can be
                     used for, but it is also available for the kernel's use
                     for its own data structures.  Among many other things,
                     it is where everything from Slab is allocated.  Bad
                     things happen when you're out of lowmem.

              LowFree %lu
                     (Starting with Linux 2.6.19, CONFIG_HIGHMEM is
                     required.)  Amount of free lowmem.

              MmapCopy %lu (since Linux 2.6.29)
                     (CONFIG_MMU is required.)  [To be documented.]

              SwapTotal %lu
                     Total amount of swap space available.

              SwapFree %lu
                     Amount of swap space that is currently unused.

              Dirty %lu
                     Memory which is waiting to get written back to the
                     disk.

              Writeback %lu
                     Memory which is actively being written back to the
                     disk.

              AnonPages %lu (since Linux 2.6.18)
                     Non-file backed pages mapped into user-space page
                     tables.

              Mapped %lu
                     Files which have been mapped into memory (with
                     mmap(2)), such as libraries.

              Shmem %lu (since Linux 2.6.32)
                     Amount of memory consumed in tmpfs(5) filesystems.

              Slab %lu
                     In-kernel data structures cache.  (See slabinfo(5).)

              SReclaimable %lu (since Linux 2.6.19)
                     Part of Slab, that might be reclaimed, such as caches.

              SUnreclaim %lu (since Linux 2.6.19)
                     Part of Slab, that cannot be reclaimed on memory pres‐
                     sure.

              KernelStack %lu (since Linux 2.6.32)
                     Amount of memory allocated to kernel stacks.

              PageTables %lu (since Linux 2.6.18)
                     Amount of memory dedicated to the lowest level of page
                     tables.

              Quicklists %lu (since Linux 2.6.27)
                     (CONFIG_QUICKLIST is required.)  [To be documented.]

              NFS_Unstable %lu (since Linux 2.6.18)
                     NFS pages sent to the server, but not yet committed to
                     stable storage.

              Bounce %lu (since Linux 2.6.18)
                     Memory used for block device "bounce buffers".

              WritebackTmp %lu (since Linux 2.6.26)
                     Memory used by FUSE for temporary writeback buffers.

              CommitLimit %lu (since Linux 2.6.10)
                     This is the total amount of memory currently available
                     to be allocated on the system, expressed in kilobytes.
                     This limit is adhered to only if strict overcommit
                     accounting is enabled (mode 2 in /proc/sys/vm/overcom‐
                     mit_memory).  The limit is calculated according to the
                     formula described under /proc/sys/vm/overcommit_memory.
                     For further details, see the kernel source file Docu‐
                     mentation/vm/overcommit-accounting.

              Committed_AS %lu
                     The amount of memory presently allocated on the system.
                     The committed memory is a sum of all of the memory
                     which has been allocated by processes, even if it has
                     not been "used" by them as of yet.  A process which
                     allocates 1GB of memory (using malloc(3) or similar),
                     but touches only 300MB of that memory will show up as
                     using only 300MB of memory even if it has the address
                     space allocated for the entire 1GB.

                     This 1GB is memory which has been "committed" to by the
                     VM and can be used at any time by the allocating appli‐
                     cation.  With strict overcommit enabled on the system
                     (mode 2 in /proc/sys/vm/overcommit_memory), allocations
                     which would exceed the CommitLimit will not be permit‐
                     ted.  This is useful if one needs to guarantee that
                     processes will not fail due to lack of memory once that
                     memory has been successfully allocated.

              VmallocTotal %lu
                     Total size of vmalloc memory area.

              VmallocUsed %lu
                     Amount of vmalloc area which is used.

              VmallocChunk %lu
                     Largest contiguous block of vmalloc area which is free.

              HardwareCorrupted %lu (since Linux 2.6.32)
                     (CONFIG_MEMORY_FAILURE is required.)  [To be docu‐
                     mented.]

              AnonHugePages %lu (since Linux 2.6.38)
                     (CONFIG_TRANSPARENT_HUGEPAGE is required.)  Non-file
                     backed huge pages mapped into user-space page tables.

              ShmemHugePages %lu (since Linux 4.8)
                     (CONFIG_TRANSPARENT_HUGEPAGE is required.)  Memory used
                     by shared memory (shmem) and tmpfs(5) allocated with
                     huge pages

              ShmemPmdMapped %lu (since Linux 4.8)
                     (CONFIG_TRANSPARENT_HUGEPAGE is required.)  Shared mem‐
                     ory mapped into user space with huge pages.

              CmaTotal %lu (since Linux 3.1)
                     Total CMA (Contiguous Memory Allocator) pages.  (CON‐
                     FIG_CMA is required.)

              CmaFree %lu (since Linux 3.1)
                     Free CMA (Contiguous Memory Allocator) pages.  (CON‐
                     FIG_CMA is required.)

              HugePages_Total %lu
                     (CONFIG_HUGETLB_PAGE is required.)  The size of the
                     pool of huge pages.

              HugePages_Free %lu
                     (CONFIG_HUGETLB_PAGE is required.)  The number of huge
                     pages in the pool that are not yet allocated.

              HugePages_Rsvd %lu (since Linux 2.6.17)
                     (CONFIG_HUGETLB_PAGE is required.)  This is the number
                     of huge pages for which a commitment to allocate from
                     the pool has been made, but no allocation has yet been
                     made.  These reserved huge pages guarantee that an
                     application will be able to allocate a huge page from
                     the pool of huge pages at fault time.

              HugePages_Surp %lu (since Linux 2.6.24)
                     (CONFIG_HUGETLB_PAGE is required.)  This is the number
                     of huge pages in the pool above the value in
                     /proc/sys/vm/nr_hugepages.  The maximum number of sur‐
                     plus huge pages is controlled by /proc/sys/vm/nr_over‐
                     commit_hugepages.

              Hugepagesize %lu
                     (CONFIG_HUGETLB_PAGE is required.)  The size of huge
                     pages.

              DirectMap4k %lu (since Linux 2.6.27)
                     Number of bytes of RAM linearly mapped by kernel in 4kB
                     pages.  (x86.)

              DirectMap4M %lu (since Linux 2.6.27)
                     Number of bytes of RAM linearly mapped by kernel in 4MB
                     pages.  (x86 with CONFIG_X86_64 or CONFIG_X86_PAE
                     enabled.)

              DirectMap2M %lu (since Linux 2.6.27)
                     Number of bytes of RAM linearly mapped by kernel in 2MB
                     pages.  (x86 with neither CONFIG_X86_64 nor CON‐
                     FIG_X86_PAE enabled.)

              DirectMap1G %lu (since Linux 2.6.27)
                     (x86 with CONFIG_X86_64 and CONFIG_X86_DIRECT_GBPAGES
                     enabled.)

       /proc/modules
              A text list of the modules that have been loaded by the sys‐
              tem.  See also lsmod(8).

       /proc/mounts
              Before kernel 2.4.19, this file was a list of all the filesys‐
              tems currently mounted on the system.  With the introduction
              of per-process mount namespaces in Linux 2.4.19 (see
              mount_namespaces(7)), this file became a link to
              /proc/self/mounts, which lists the mount points of the
              process's own mount namespace.  The format of this file is
              documented in fstab(5).

       /proc/mtrr
              Memory Type Range Registers.  See the Linux kernel source file
              Documentation/x86/mtrr.txt (or Documentation/mtrr.txt before
              Linux 2.6.28) for details.

       /proc/net
              This directory contains various files and subdirectories con‐
              taining information about the networking layer.  The files
              contain ASCII structures and are, therefore, readable with
              cat(1).  However, the standard netstat(8) suite provides much
              cleaner access to these files.

              With the advent of network namespaces, various information
              relating to the network stack is virtualized (see
              namespaces(7)).  Thus, since Linux 2.6.25, /proc/net is a sym‐
              bolic link to the directory /proc/self/net, which contains the
              same files and directories as listed below.  However, these
              files and directories now expose information for the network
              namespace of which the process is a member.

       /proc/net/arp
              This holds an ASCII readable dump of the kernel ARP table used
              for address resolutions.  It will show both dynamically
              learned and preprogrammed ARP entries.  The format is:

       IP address     HW type   Flags     HW address          Mask   Device
       192.168.0.50   0x1       0x2       00:50:BF:25:68:F3   *      eth0
       192.168.0.250  0x1       0xc       00:00:00:00:00:00   *      eth0

              Here "IP address" is the IPv4 address of the machine and the
              "HW type" is the hardware type of the address from RFC 826.
              The flags are the internal flags of the ARP structure (as
              defined in /usr/include/linux/if_arp.h) and the "HW address"
              is the data link layer mapping for that IP address if it is
              known.

       /proc/net/dev
              The dev pseudo-file contains network device status informa‐
              tion.  This gives the number of received and sent packets, the
              number of errors and collisions and other basic statistics.
              These are used by the ifconfig(8) program to report device
              status.  The format is:

 Inter-|   Receive                                                |  Transmit
  face |bytes    packets errs drop fifo frame compressed multicast|bytes    packets errs drop fifo colls carrier compressed
     lo: 2776770   11307    0    0    0     0          0         0  2776770   11307    0    0    0     0       0          0
   eth0: 1215645    2751    0    0    0     0          0         0  1782404    4324    0    0    0   427       0          0
   ppp0: 1622270    5552    1    0    0     0          0         0   354130    5669    0    0    0     0       0          0
   tap0:    7714      81    0    0    0     0          0         0     7714      81    0    0    0     0       0          0

       /proc/net/dev_mcast
              Defined in /usr/src/linux/net/core/dev_mcast.c:

                  indx interface_name  dmi_u dmi_g dmi_address
                  2    eth0            1     0     01005e000001
                  3    eth1            1     0     01005e000001
                  4    eth2            1     0     01005e000001

       /proc/net/igmp
              Internet Group Management Protocol.  Defined in
              /usr/src/linux/net/core/igmp.c.

       /proc/net/rarp
              This file uses the same format as the arp file and contains
              the current reverse mapping database used to provide rarp(8)
              reverse address lookup services.  If RARP is not configured
              into the kernel, this file will not be present.

       /proc/net/raw
              Holds a dump of the RAW socket table.  Much of the information
              is not of use apart from debugging.  The "sl" value is the
              kernel hash slot for the socket, the "local_address" is the
              local address and protocol number pair.  "St" is the internal
              status of the socket.  The "tx_queue" and "rx_queue" are the
              outgoing and incoming data queue in terms of kernel memory
              usage.  The "tr", "tm->when", and "rexmits" fields are not
              used by RAW.  The "uid" field holds the effective UID of the
              creator of the socket.

       /proc/net/snmp
              This file holds the ASCII data needed for the IP, ICMP, TCP,
              and UDP management information bases for an SNMP agent.

       /proc/net/tcp
              Holds a dump of the TCP socket table.  Much of the information
              is not of use apart from debugging.  The "sl" value is the
              kernel hash slot for the socket, the "local_address" is the
              local address and port number pair.  The "rem_address" is the
              remote address and port number pair (if connected).  "St" is
              the internal status of the socket.  The "tx_queue" and
              "rx_queue" are the outgoing and incoming data queue in terms
              of kernel memory usage.  The "tr", "tm->when", and "rexmits"
              fields hold internal information of the kernel socket state
              and are useful only for debugging.  The "uid" field holds the
              effective UID of the creator of the socket.

       /proc/net/udp
              Holds a dump of the UDP socket table.  Much of the information
              is not of use apart from debugging.  The "sl" value is the
              kernel hash slot for the socket, the "local_address" is the
              local address and port number pair.  The "rem_address" is the
              remote address and port number pair (if connected).  "St" is
              the internal status of the socket.  The "tx_queue" and
              "rx_queue" are the outgoing and incoming data queue in terms
              of kernel memory usage.  The "tr", "tm->when", and "rexmits"
              fields are not used by UDP.  The "uid" field holds the effec‐
              tive UID of the creator of the socket.  The format is:

 sl  local_address rem_address   st tx_queue rx_queue tr rexmits  tm->when uid
  1: 01642C89:0201 0C642C89:03FF 01 00000000:00000001 01:000071BA 00000000 0
  1: 00000000:0801 00000000:0000 0A 00000000:00000000 00:00000000 6F000100 0
  1: 00000000:0201 00000000:0000 0A 00000000:00000000 00:00000000 00000000 0

       /proc/net/unix
              Lists the UNIX domain sockets present within the system and
              their status.  The format is:

 Num RefCount Protocol Flags    Type St Path
  0: 00000002 00000000 00000000 0001 03
  1: 00000001 00000000 00010000 0001 01 /dev/printer

              The fields are as follows:

              Num:      the kernel table slot number.

              RefCount: the number of users of the socket.

              Protocol: currently always 0.

              Flags:    the internal kernel flags holding the status of the
                        socket.

              Type:     the socket type.  For SOCK_STREAM sockets, this is
                        0001; for SOCK_DGRAM sockets, it is 0002; and for
                        SOCK_SEQPACKET sockets, it is 0005.

              St:       the internal state of the socket.

              Path:     the bound path (if any) of the socket.  Sockets in
                        the abstract namespace are included in the list, and
                        are shown with a Path that commences with the char‐
                        acter '@'.

       /proc/net/netfilter/nfnetlink_queue
              This file contains information about netfilter user-space
              queueing, if used.  Each line represents a queue.  Queues that
              have not been subscribed to by user space are not shown.

                     1   4207     0  2 65535     0     0        0  1
                    (1)   (2)    (3)(4)  (5)    (6)   (7)      (8)

              The fields in each line are:

              (1)  The ID of the queue.  This matches what is specified in
                   the --queue-num or --queue-balance options to the
                   iptables(8) NFQUEUE target.  See iptables-extensions(8)
                   for more information.

              (2)  The netlink port ID subscribed to the queue.

              (3)  The number of packets currently queued and waiting to be
                   processed by the application.

              (4)  The copy mode of the queue.  It is either 1 (metadata
                   only) or 2 (also copy payload data to user space).

              (5)  Copy range; that is, how many bytes of packet payload
                   should be copied to user space at most.

              (6)  queue dropped.  Number of packets that had to be dropped
                   by the kernel because too many packets are already wait‐
                   ing for user space to send back the mandatory accept/drop
                   verdicts.

              (7)  queue user dropped.  Number of packets that were dropped
                   within the netlink subsystem.  Such drops usually happen
                   when the corresponding socket buffer is full; that is,
                   user space is not able to read messages fast enough.

              (8)  sequence number.  Every queued packet is associated with
                   a (32-bit) monotonically-increasing sequence number.
                   This shows the ID of the most recent packet queued.

              The last number exists only for compatibility reasons and is
              always 1.

       /proc/partitions
              Contains the major and minor numbers of each partition as well
              as the number of 1024-byte blocks and the partition name.

       /proc/pci
              This is a listing of all PCI devices found during kernel ini‐
              tialization and their configuration.

              This file has been deprecated in favor of a new /proc inter‐
              face for PCI (/proc/bus/pci).  It became optional in Linux 2.2
              (available with CONFIG_PCI_OLD_PROC set at kernel compila‐
              tion).  It became once more nonoptionally enabled in Linux
              2.4.  Next, it was deprecated in Linux 2.6 (still available
              with CONFIG_PCI_LEGACY_PROC set), and finally removed alto‐
              gether since Linux 2.6.17.

       /proc/profile (since Linux 2.4)
              This file is present only if the kernel was booted with the
              profile=1 command-line option.  It exposes kernel profiling
              information in a binary format for use by readprofile(1).
              Writing (e.g., an empty string) to this file resets the pro‐
              filing counters; on some architectures, writing a binary inte‐
              ger "profiling multiplier" of size sizeof(int) sets the pro‐
              filing interrupt frequency.

       /proc/scsi
              A directory with the scsi mid-level pseudo-file and various
              SCSI low-level driver directories, which contain a file for
              each SCSI host in this system, all of which give the status of
              some part of the SCSI IO subsystem.  These files contain ASCII
              structures and are, therefore, readable with cat(1).

              You can also write to some of the files to reconfigure the
              subsystem or switch certain features on or off.

       /proc/scsi/scsi
              This is a listing of all SCSI devices known to the kernel.
              The listing is similar to the one seen during bootup.  scsi
              currently supports only the add-single-device command which
              allows root to add a hotplugged device to the list of known
              devices.

              The command

                  echo 'scsi add-single-device 1 0 5 0' > /proc/scsi/scsi

              will cause host scsi1 to scan on SCSI channel 0 for a device
              on ID 5 LUN 0.  If there is already a device known on this
              address or the address is invalid, an error will be returned.

       /proc/scsi/[drivername]
              [drivername] can currently be NCR53c7xx, aha152x, aha1542,
              aha1740, aic7xxx, buslogic, eata_dma, eata_pio, fdomain,
              in2000, pas16, qlogic, scsi_debug, seagate, t128, u15-24f,
              ultrastore, or wd7000.  These directories show up for all
              drivers that registered at least one SCSI HBA.  Every direc‐
              tory contains one file per registered host.  Every host-file
              is named after the number the host was assigned during ini‐
              tialization.

              Reading these files will usually show driver and host configu‐
              ration, statistics, and so on.

              Writing to these files allows different things on different
              hosts.  For example, with the latency and nolatency commands,
              root can switch on and off command latency measurement code in
              the eata_dma driver.  With the lockup and unlock commands,
              root can control bus lockups simulated by the scsi_debug
              driver.

       /proc/self
              This directory refers to the process accessing the /proc
              filesystem, and is identical to the /proc directory named by
              the process ID of the same process.

       /proc/slabinfo
              Information about kernel caches.  See slabinfo(5) for details.

       /proc/stat
              kernel/system statistics.  Varies with architecture.  Common
              entries include:

              cpu 10132153 290696 3084719 46828483 16683 0 25195 0 175628 0
              cpu0 1393280 32966 572056 13343292 6130 0 17875 0 23933 0
                     The amount of time, measured in units of USER_HZ
                     (1/100ths of a second on most architectures, use
                     sysconf(_SC_CLK_TCK) to obtain the right value), that
                     the system ("cpu" line) or the specific CPU ("cpuN"
                     line) spent in various states:

                     user   (1) Time spent in user mode.

                     nice   (2) Time spent in user mode with low priority
                            (nice).

                     system (3) Time spent in system mode.

                     idle   (4) Time spent in the idle task.  This value
                            should be USER_HZ times the second entry in the
                            /proc/uptime pseudo-file.

                     iowait (since Linux 2.5.41)
                            (5) Time waiting for I/O to complete.  This
                            value is not reliable, for the following rea‐
                            sons:

                            1. The CPU will not wait for I/O to complete;
                               iowait is the time that a task is waiting for
                               I/O to complete.  When a CPU goes into idle
                               state for outstanding task I/O, another task
                               will be scheduled on this CPU.

                            2. On a multi-core CPU, the task waiting for I/O
                               to complete is not running on any CPU, so the
                               iowait of each CPU is difficult to calculate.

                            3. The value in this field may decrease in cer‐
                               tain conditions.

                     irq (since Linux 2.6.0-test4)
                            (6) Time servicing interrupts.

                     softirq (since Linux 2.6.0-test4)
                            (7) Time servicing softirqs.

                     steal (since Linux 2.6.11)
                            (8) Stolen time, which is the time spent in
                            other operating systems when running in a virtu‐
                            alized environment

                     guest (since Linux 2.6.24)
                            (9) Time spent running a virtual CPU for guest
                            operating systems under the control of the Linux
                            kernel.

                     guest_nice (since Linux 2.6.33)
                            (10) Time spent running a niced guest (virtual
                            CPU for guest operating systems under the con‐
                            trol of the Linux kernel).

              page 5741 1808
                     The number of pages the system paged in and the number
                     that were paged out (from disk).

              swap 1 0
                     The number of swap pages that have been brought in and
                     out.

              intr 1462898
                     This line shows counts of interrupts serviced since
                     boot time, for each of the possible system interrupts.
                     The first column is the total of all interrupts ser‐
                     viced including unnumbered architecture specific inter‐
                     rupts; each subsequent column is the total for that
                     particular numbered interrupt.  Unnumbered interrupts
                     are not shown, only summed into the total.

              disk_io: (2,0):(31,30,5764,1,2) (3,0):...
                     (major,disk_idx):(noinfo, read_io_ops, blks_read,
                     write_io_ops, blks_written)
                     (Linux 2.4 only)

              ctxt 115315
                     The number of context switches that the system under‐
                     went.

              btime 769041601
                     boot time, in seconds since the Epoch, 1970-01-01
                     00:00:00 +0000 (UTC).

              processes 86031
                     Number of forks since boot.

              procs_running 6
                     Number of processes in runnable state.  (Linux 2.5.45
                     onward.)

              procs_blocked 2
                     Number of processes blocked waiting for I/O to com‐
                     plete.  (Linux 2.5.45 onward.)

              softirq 229245889 94 60001584 13619 5175704 2471304 28
              51212741 59130143 0 51240672
                     This line shows the number of softirq for all CPUs.
                     The first column is the total of all softirqs and each
                     subsequent column is the total for particular softirq.
                     (Linux 2.6.31 onward.)

       /proc/swaps
              Swap areas in use.  See also swapon(8).

       /proc/sys
              This directory (present since 1.3.57) contains a number of
              files and subdirectories corresponding to kernel variables.
              These variables can be read and sometimes modified using the
              /proc filesystem, and the (deprecated) sysctl(2) system call.

              String values may be terminated by either '\0' or '\n'.

              Integer and long values may be written either in decimal or in
              hexadecimal notation (e.g. 0x3FFF).  When writing multiple
              integer or long values, these may be separated by any of the
              following whitespace characters: ' ', '\t', or '\n'.  Using
              other separators leads to the error EINVAL.

       /proc/sys/abi (since Linux 2.4.10)
              This directory may contain files with application binary
              information.  See the Linux kernel source file Documenta‐
              tion/sysctl/abi.txt for more information.

       /proc/sys/debug
              This directory may be empty.

       /proc/sys/dev
              This directory contains device-specific information (e.g.,
              dev/cdrom/info).  On some systems, it may be empty.

       /proc/sys/fs
              This directory contains the files and subdirectories for ker‐
              nel variables related to filesystems.

       /proc/sys/fs/binfmt_misc
              Documentation for files in this directory can be found in the
              Linux kernel source in the file Documentation/admin-
              guide/binfmt-misc.rst (or in Documentation/binfmt_misc.txt on
              older kernels).

       /proc/sys/fs/dentry-state (since Linux 2.2)
              This file contains information about the status of the direc‐
              tory cache (dcache).  The file contains six numbers, nr_den‐
              try, nr_unused, age_limit (age in seconds), want_pages (pages
              requested by system) and two dummy values.

              * nr_dentry is the number of allocated dentries (dcache
                entries).  This field is unused in Linux 2.2.

              * nr_unused is the number of unused dentries.

              * age_limit is the age in seconds after which dcache entries
                can be reclaimed when memory is short.

              * want_pages is nonzero when the kernel has called
                shrink_dcache_pages() and the dcache isn't pruned yet.

       /proc/sys/fs/dir-notify-enable
              This file can be used to disable or enable the dnotify inter‐
              face described in fcntl(2) on a system-wide basis.  A value of
              0 in this file disables the interface, and a value of 1
              enables it.

       /proc/sys/fs/dquot-max
              This file shows the maximum number of cached disk quota
              entries.  On some (2.4) systems, it is not present.  If the
              number of free cached disk quota entries is very low and you
              have some awesome number of simultaneous system users, you
              might want to raise the limit.

       /proc/sys/fs/dquot-nr
              This file shows the number of allocated disk quota entries and
              the number of free disk quota entries.

       /proc/sys/fs/epoll (since Linux 2.6.28)
              This directory contains the file max_user_watches, which can
              be used to limit the amount of kernel memory consumed by the
              epoll interface.  For further details, see epoll(7).

       /proc/sys/fs/file-max
              This file defines a system-wide limit on the number of open
              files for all processes.  System calls that fail when encoun‐
              tering this limit fail with the error ENFILE.  (See also
              setrlimit(2), which can be used by a process to set the per-
              process limit, RLIMIT_NOFILE, on the number of files it may
              open.)  If you get lots of error messages in the kernel log
              about running out of file handles (look for "VFS: file-max
              limit <number> reached"), try increasing this value:

                  echo 100000 > /proc/sys/fs/file-max

              Privileged processes (CAP_SYS_ADMIN) can override the file-max
              limit.

       /proc/sys/fs/file-nr
              This (read-only) file contains three numbers: the number of
              allocated file handles (i.e., the number of files presently
              opened); the number of free file handles; and the maximum num‐
              ber of file handles (i.e., the same value as
              /proc/sys/fs/file-max).  If the number of allocated file han‐
              dles is close to the maximum, you should consider increasing
              the maximum.  Before Linux 2.6, the kernel allocated file han‐
              dles dynamically, but it didn't free them again.  Instead the
              free file handles were kept in a list for reallocation; the
              "free file handles" value indicates the size of that list.  A
              large number of free file handles indicates that there was a
              past peak in the usage of open file handles.  Since Linux 2.6,
              the kernel does deallocate freed file handles, and the "free
              file handles" value is always zero.

       /proc/sys/fs/inode-max (only present until Linux 2.2)
              This file contains the maximum number of in-memory inodes.
              This value should be 3–4 times larger than the value in file-
              max, since stdin, stdout and network sockets also need an
              inode to handle them.  When you regularly run out of inodes,
              you need to increase this value.

              Starting with Linux 2.4, there is no longer a static limit on
              the number of inodes, and this file is removed.

       /proc/sys/fs/inode-nr
              This file contains the first two values from inode-state.

       /proc/sys/fs/inode-state
              This file contains seven numbers: nr_inodes, nr_free_inodes,
              preshrink, and four dummy values (always zero).

              nr_inodes is the number of inodes the system has allocated.
              nr_free_inodes represents the number of free inodes.

              preshrink is nonzero when the nr_inodes > inode-max and the
              system needs to prune the inode list instead of allocating
              more; since Linux 2.4, this field is a dummy value (always
              zero).

       /proc/sys/fs/inotify (since Linux 2.6.13)
              This directory contains files max_queued_events,
              max_user_instances, and max_user_watches, that can be used to
              limit the amount of kernel memory consumed by the inotify
              interface.  For further details, see inotify(7).

       /proc/sys/fs/lease-break-time
              This file specifies the grace period that the kernel grants to
              a process holding a file lease (fcntl(2)) after it has sent a
              signal to that process notifying it that another process is
              waiting to open the file.  If the lease holder does not remove
              or downgrade the lease within this grace period, the kernel
              forcibly breaks the lease.

       /proc/sys/fs/leases-enable
              This file can be used to enable or disable file leases
              (fcntl(2)) on a system-wide basis.  If this file contains the
              value 0, leases are disabled.  A nonzero value enables leases.

       /proc/sys/fs/mount-max (since Linux 4.9)
              The value in this file specifies the maximum number of mounts
              that may exist in a mount namespace.  The default value in
              this file is 100,000.

       /proc/sys/fs/mqueue (since Linux 2.6.6)
              This directory contains files msg_max, msgsize_max, and
              queues_max, controlling the resources used by POSIX message
              queues.  See mq_overview(7) for details.

       /proc/sys/fs/nr_open (since Linux 2.6.25)
              This file imposes ceiling on the value to which the
              RLIMIT_NOFILE resource limit can be raised (see getrlimit(2)).
              This ceiling is enforced for both unprivileged and privileged
              process.  The default value in this file is 1048576.  (Before
              Linux 2.6.25, the ceiling for RLIMIT_NOFILE was hard-coded to
              the same value.)

       /proc/sys/fs/overflowgid and /proc/sys/fs/overflowuid
              These files allow you to change the value of the fixed UID and
              GID.  The default is 65534.  Some filesystems support only
              16-bit UIDs and GIDs, although in Linux UIDs and GIDs are 32
              bits.  When one of these filesystems is mounted with writes
              enabled, any UID or GID that would exceed 65535 is translated
              to the overflow value before being written to disk.

       /proc/sys/fs/pipe-max-size (since Linux 2.6.35)
              See pipe(7).

       /proc/sys/fs/pipe-user-pages-hard (since Linux 4.5)
              See pipe(7).

       /proc/sys/fs/pipe-user-pages-soft (since Linux 4.5)
              See pipe(7).

       /proc/sys/fs/protected_hardlinks (since Linux 3.6)
              When the value in this file is 0, no restrictions are placed
              on the creation of hard links (i.e., this is the historical
              behavior before Linux 3.6).  When the value in this file is 1,
              a hard link can be created to a target file only if one of the
              following conditions is true:

              *  The calling process has the CAP_FOWNER capability in its
                 user namespace and the file UID has a mapping in the names‐
                 pace.

              *  The filesystem UID of the process creating the link matches
                 the owner (UID) of the target file (as described in
                 credentials(7), a process's filesystem UID is normally the
                 same as its effective UID).

              *  All of the following conditions are true:

                  ·  the target is a regular file;

                  ·  the target file does not have its set-user-ID mode bit
                     enabled;

                  ·  the target file does not have both its set-group-ID and
                     group-executable mode bits enabled; and

                  ·  the caller has permission to read and write the target
                     file (either via the file's permissions mask or because
                     it has suitable capabilities).

              The default value in this file is 0.  Setting the value to 1
              prevents a longstanding class of security issues caused by
              hard-link-based time-of-check, time-of-use races, most com‐
              monly seen in world-writable directories such as /tmp.  The
              common method of exploiting this flaw is to cross privilege
              boundaries when following a given hard link (i.e., a root
              process follows a hard link created by another user).  Addi‐
              tionally, on systems without separated partitions, this stops
              unauthorized users from "pinning" vulnerable set-user-ID and
              set-group-ID files against being upgraded by the administra‐
              tor, or linking to special files.

       /proc/sys/fs/protected_symlinks (since Linux 3.6)
              When the value in this file is 0, no restrictions are placed
              on following symbolic links (i.e., this is the historical
              behavior before Linux 3.6).  When the value in this file is 1,
              symbolic links are followed only in the following circum‐
              stances:

              *  the filesystem UID of the process following the link
                 matches the owner (UID) of the symbolic link (as described
                 in credentials(7), a process's filesystem UID is normally
                 the same as its effective UID);

              *  the link is not in a sticky world-writable directory; or

              *  the symbolic link and its parent directory have the same
                 owner (UID)

              A system call that fails to follow a symbolic link because of
              the above restrictions returns the error EACCES in errno.

              The default value in this file is 0.  Setting the value to 1
              avoids a longstanding class of security issues based on time-
              of-check, time-of-use races when accessing symbolic links.

       /proc/sys/fs/suid_dumpable (since Linux 2.6.13)
              The value in this file is assigned to a process's "dumpable"
              flag in the circumstances described in prctl(2).  In effect,
              the value in this file determines whether core dump files are
              produced for set-user-ID or otherwise protected/tainted bina‐
              ries.  The "dumpable" setting also affects the ownership of
              files in a process's /proc/[pid] directory, as described
              above.

              Three different integer values can be specified:

              0 (default)
                     This provides the traditional (pre-Linux 2.6.13) behav‐
                     ior.  A core dump will not be produced for a process
                     which has changed credentials (by calling seteuid(2),
                     setgid(2), or similar, or by executing a set-user-ID or
                     set-group-ID program) or whose binary does not have
                     read permission enabled.

              1 ("debug")
                     All processes dump core when possible.  (Reasons why a
                     process might nevertheless not dump core are described
                     in core(5).)  The core dump is owned by the filesystem
                     user ID of the dumping process and no security is
                     applied.  This is intended for system debugging situa‐
                     tions only: this mode is insecure because it allows
                     unprivileged users to examine the memory contents of
                     privileged processes.

              2 ("suidsafe")
                     Any binary which normally would not be dumped (see "0"
                     above) is dumped readable by root only.  This allows
                     the user to remove the core dump file but not to read
                     it.  For security reasons core dumps in this mode will
                     not overwrite one another or other files.  This mode is
                     appropriate when administrators are attempting to debug
                     problems in a normal environment.

                     Additionally, since Linux 3.6, /proc/sys/ker‐
                     nel/core_pattern must either be an absolute pathname or
                     a pipe command, as detailed in core(5).  Warnings will
                     be written to the kernel log if core_pattern does not
                     follow these rules, and no core dump will be produced.

              For details of the effect of a process's "dumpable" setting on
              ptrace access mode checking, see ptrace(2).

       /proc/sys/fs/super-max
              This file controls the maximum number of superblocks, and thus
              the maximum number of mounted filesystems the kernel can have.
              You need increase only super-max if you need to mount more
              filesystems than the current value in super-max allows you to.

       /proc/sys/fs/super-nr
              This file contains the number of filesystems currently
              mounted.

       /proc/sys/kernel
              This directory contains files controlling a range of kernel
              parameters, as described below.

       /proc/sys/kernel/acct
              This file contains three numbers: highwater, lowwater, and
              frequency.  If BSD-style process accounting is enabled, these
              values control its behavior.  If free space on filesystem
              where the log lives goes below lowwater percent, accounting
              suspends.  If free space gets above highwater percent,
              accounting resumes.  frequency determines how often the kernel
              checks the amount of free space (value is in seconds).
              Default values are 4, 2 and 30.  That is, suspend accounting
              if 2% or less space is free; resume it if 4% or more space is
              free; consider information about amount of free space valid
              for 30 seconds.

       /proc/sys/kernel/auto_msgmni (Linux 2.6.27 to 3.18)
              From Linux 2.6.27 to 3.18, this file was used to control
              recomputing of the value in /proc/sys/kernel/msgmni upon the
              addition or removal of memory or upon IPC namespace cre‐
              ation/removal.  Echoing "1" into this file enabled msgmni
              automatic recomputing (and triggered a recomputation of msgmni
              based on the current amount of available memory and number of
              IPC namespaces).  Echoing "0" disabled automatic recomputing.
              (Automatic recomputing was also disabled if a value was
              explicitly assigned to /proc/sys/kernel/msgmni.)  The default
              value in auto_msgmni was 1.

              Since Linux 3.19, the content of this file has no effect
              (because msgmni defaults to near the maximum value possible),
              and reads from this file always return the value "0".

       /proc/sys/kernel/cap_last_cap (since Linux 3.2)
              See capabilities(7).

       /proc/sys/kernel/cap-bound (from Linux 2.2 to 2.6.24)
              This file holds the value of the kernel capability bounding
              set (expressed as a signed decimal number).  This set is ANDed
              against the capabilities permitted to a process during
              execve(2).  Starting with Linux 2.6.25, the system-wide capa‐
              bility bounding set disappeared, and was replaced by a per-
              thread bounding set; see capabilities(7).

       /proc/sys/kernel/core_pattern
              See core(5).

       /proc/sys/kernel/core_pipe_limit
              See core(5).

       /proc/sys/kernel/core_uses_pid
              See core(5).

       /proc/sys/kernel/ctrl-alt-del
              This file controls the handling of Ctrl-Alt-Del from the key‐
              board.  When the value in this file is 0, Ctrl-Alt-Del is
              trapped and sent to the init(1) program to handle a graceful
              restart.  When the value is greater than zero, Linux's reac‐
              tion to a Vulcan Nerve Pinch (tm) will be an immediate reboot,
              without even syncing its dirty buffers.  Note: when a program
              (like dosemu) has the keyboard in "raw" mode, the ctrl-alt-del
              is intercepted by the program before it ever reaches the ker‐
              nel tty layer, and it's up to the program to decide what to do
              with it.

       /proc/sys/kernel/dmesg_restrict (since Linux 2.6.37)
              The value in this file determines who can see kernel syslog
              contents.  A value of 0 in this file imposes no restrictions.
              If the value is 1, only privileged users can read the kernel
              syslog.  (See syslog(2) for more details.)  Since Linux 3.4,
              only users with the CAP_SYS_ADMIN capability may change the
              value in this file.

       /proc/sys/kernel/domainname and /proc/sys/kernel/hostname
              can be used to set the NIS/YP domainname and the hostname of
              your box in exactly the same way as the commands domainname(1)
              and hostname(1), that is:

                  # echo 'darkstar' > /proc/sys/kernel/hostname
                  # echo 'mydomain' > /proc/sys/kernel/domainname

              has the same effect as

                  # hostname 'darkstar'
                  # domainname 'mydomain'

              Note, however, that the classic darkstar.frop.org has the
              hostname "darkstar" and DNS (Internet Domain Name Server)
              domainname "frop.org", not to be confused with the NIS (Net‐
              work Information Service) or YP (Yellow Pages) domainname.
              These two domain names are in general different.  For a
              detailed discussion see the hostname(1) man page.

       /proc/sys/kernel/hotplug
              This file contains the path for the hotplug policy agent.  The
              default value in this file is /sbin/hotplug.

       /proc/sys/kernel/htab-reclaim (before Linux 2.4.9.2)
              (PowerPC only) If this file is set to a nonzero value, the
              PowerPC htab (see kernel file Documentation/pow‐
              erpc/ppc_htab.txt) is pruned each time the system hits the
              idle loop.

       /proc/sys/kernel/keys/*
              This directory contains various files that define parameters
              and limits for the key-management facility.  These files are
              described in keyrings(7).

       /proc/sys/kernel/kptr_restrict (since Linux 2.6.38)
              The value in this file determines whether kernel addresses are
              exposed via /proc files and other interfaces.  A value of 0 in
              this file imposes no restrictions.  If the value is 1, kernel
              pointers printed using the %pK format specifier will be
              replaced with zeros unless the user has the CAP_SYSLOG capa‐
              bility.  If the value is 2, kernel pointers printed using the
              %pK format specifier will be replaced with zeros regardless of
              the user's capabilities.  The initial default value for this
              file was 1, but the default was changed to 0 in Linux 2.6.39.
              Since Linux 3.4, only users with the CAP_SYS_ADMIN capability
              can change the value in this file.

       /proc/sys/kernel/l2cr
              (PowerPC only) This file contains a flag that controls the L2
              cache of G3 processor boards.  If 0, the cache is disabled.
              Enabled if nonzero.

       /proc/sys/kernel/modprobe
              This file contains the path for the kernel module loader.  The
              default value is /sbin/modprobe.  The file is present only if
              the kernel is built with the CONFIG_MODULES (CONFIG_KMOD in
              Linux 2.6.26 and earlier) option enabled.  It is described by
              the Linux kernel source file Documentation/kmod.txt (present
              only in kernel 2.4 and earlier).

       /proc/sys/kernel/modules_disabled (since Linux 2.6.31)
              A toggle value indicating if modules are allowed to be loaded
              in an otherwise modular kernel.  This toggle defaults to off
              (0), but can be set true (1).  Once true, modules can be nei‐
              ther loaded nor unloaded, and the toggle cannot be set back to
              false.  The file is present only if the kernel is built with
              the CONFIG_MODULES option enabled.

       /proc/sys/kernel/msgmax (since Linux 2.2)
              This file defines a system-wide limit specifying the maximum
              number of bytes in a single message written on a System V mes‐
              sage queue.

       /proc/sys/kernel/msgmni (since Linux 2.4)
              This file defines the system-wide limit on the number of mes‐
              sage queue identifiers.  See also /proc/sys/ker‐
              nel/auto_msgmni.

       /proc/sys/kernel/msgmnb (since Linux 2.2)
              This file defines a system-wide parameter used to initialize
              the msg_qbytes setting for subsequently created message
              queues.  The msg_qbytes setting specifies the maximum number
              of bytes that may be written to the message queue.

       /proc/sys/kernel/ngroups_max (since Linux 2.6.4)
              This is a read-only file that displays the upper limit on the
              number of a process's group memberships.

       /proc/sys/kernel/ns_last_pid (since Linux 3.3)
              See pid_namespaces(7).

       /proc/sys/kernel/ostype and /proc/sys/kernel/osrelease
              These files give substrings of /proc/version.

       /proc/sys/kernel/overflowgid and /proc/sys/kernel/overflowuid
              These files duplicate the files /proc/sys/fs/overflowgid and
              /proc/sys/fs/overflowuid.

       /proc/sys/kernel/panic
              This file gives read/write access to the kernel variable
              panic_timeout.  If this is zero, the kernel will loop on a
              panic; if nonzero, it indicates that the kernel should autore‐
              boot after this number of seconds.  When you use the software
              watchdog device driver, the recommended setting is 60.

       /proc/sys/kernel/panic_on_oops (since Linux 2.5.68)
              This file controls the kernel's behavior when an oops or BUG
              is encountered.  If this file contains 0, then the system
              tries to continue operation.  If it contains 1, then the sys‐
              tem delays a few seconds (to give klogd time to record the
              oops output) and then panics.  If the /proc/sys/kernel/panic
              file is also nonzero, then the machine will be rebooted.

       /proc/sys/kernel/pid_max (since Linux 2.5.34)
              This file specifies the value at which PIDs wrap around (i.e.,
              the value in this file is one greater than the maximum PID).
              PIDs greater than this value are not allocated; thus, the
              value in this file also acts as a system-wide limit on the
              total number of processes and threads.  The default value for
              this file, 32768, results in the same range of PIDs as on ear‐
              lier kernels.  On 32-bit platforms, 32768 is the maximum value
              for pid_max.  On 64-bit systems, pid_max can be set to any
              value up to 2^22 (PID_MAX_LIMIT, approximately 4 million).

       /proc/sys/kernel/powersave-nap (PowerPC only)
              This file contains a flag.  If set, Linux-PPC will use the
              "nap" mode of powersaving, otherwise the "doze" mode will be
              used.

       /proc/sys/kernel/printk
              See syslog(2).

       /proc/sys/kernel/pty (since Linux 2.6.4)
              This directory contains two files relating to the number of
              UNIX 98 pseudoterminals (see pts(4)) on the system.

       /proc/sys/kernel/pty/max
              This file defines the maximum number of pseudoterminals.

       /proc/sys/kernel/pty/nr
              This read-only file indicates how many pseudoterminals are
              currently in use.

       /proc/sys/kernel/random
              This directory contains various parameters controlling the
              operation of the file /dev/random.  See random(4) for further
              information.

       /proc/sys/kernel/random/uuid (since Linux 2.4)
              Each read from this read-only file returns a randomly gener‐
              ated 128-bit UUID, as a string in the standard UUID format.

       /proc/sys/kernel/randomize_va_space (since Linux 2.6.12)
              Select the address space layout randomization (ASLR) policy
              for the system (on architectures that support ASLR).  Three
              values are supported for this file:

              0  Turn ASLR off.  This is the default for architectures that
                 don't support ASLR, and when the kernel is booted with the
                 norandmaps parameter.

              1  Make the addresses of mmap(2) allocations, the stack, and
                 the VDSO page randomized.  Among other things, this means
                 that shared libraries will be loaded at randomized
                 addresses.  The text segment of PIE-linked binaries will
                 also be loaded at a randomized address.  This value is the
                 default if the kernel was configured with CONFIG_COM‐
                 PAT_BRK.

              2  (Since Linux 2.6.25) Also support heap randomization.  This
                 value is the default if the kernel was not configured with
                 CONFIG_COMPAT_BRK.

       /proc/sys/kernel/real-root-dev
              This file is documented in the Linux kernel source file Docu‐
              mentation/admin-guide/initrd.rst (or Documentation/initrd.txt
              before Linux 4.10).

       /proc/sys/kernel/reboot-cmd (Sparc only)
              This file seems to be a way to give an argument to the SPARC
              ROM/Flash boot loader.  Maybe to tell it what to do after
              rebooting?

       /proc/sys/kernel/rtsig-max
              (Only in kernels up to and including 2.6.7; see setrlimit(2))
              This file can be used to tune the maximum number of POSIX
              real-time (queued) signals that can be outstanding in the sys‐
              tem.

       /proc/sys/kernel/rtsig-nr
              (Only in kernels up to and including 2.6.7.)  This file shows
              the number of POSIX real-time signals currently queued.

       /proc/[pid]/sched_autogroup_enabled (since Linux 2.6.38)
              See sched(7).

       /proc/sys/kernel/sched_child_runs_first (since Linux 2.6.23)
              If this file contains the value zero, then, after a fork(2),
              the parent is first scheduled on the CPU.  If the file con‐
              tains a nonzero value, then the child is scheduled first on
              the CPU.  (Of course, on a multiprocessor system, the parent
              and the child might both immediately be scheduled on a CPU.)

       /proc/sys/kernel/sched_rr_timeslice_ms (since Linux 3.9)
              See sched_rr_get_interval(2).

       /proc/sys/kernel/sched_rt_period_us (since Linux 2.6.25)
              See sched(7).

       /proc/sys/kernel/sched_rt_runtime_us (since Linux 2.6.25)
              See sched(7).

       /proc/sys/kernel/seccomp (since Linux 4.14)
              This directory provides additional seccomp information and
              configuration.  See seccomp(2) for further details.

       /proc/sys/kernel/sem (since Linux 2.4)
              This file contains 4 numbers defining limits for System V IPC
              semaphores.  These fields are, in order:

              SEMMSL  The maximum semaphores per semaphore set.

              SEMMNS  A system-wide limit on the number of semaphores in all
                      semaphore sets.

              SEMOPM  The maximum number of operations that may be specified
                      in a semop(2) call.

              SEMMNI  A system-wide limit on the maximum number of semaphore
                      identifiers.

       /proc/sys/kernel/sg-big-buff
              This file shows the size of the generic SCSI device (sg) buf‐
              fer.  You can't tune it just yet, but you could change it at
              compile time by editing include/scsi/sg.h and changing the
              value of SG_BIG_BUFF.  However, there shouldn't be any reason
              to change this value.

       /proc/sys/kernel/shm_rmid_forced (since Linux 3.1)
              If this file is set to 1, all System V shared memory segments
              will be marked for destruction as soon as the number of
              attached processes falls to zero; in other words, it is no
              longer possible to create shared memory segments that exist
              independently of any attached process.

              The effect is as though a shmctl(2) IPC_RMID is performed on
              all existing segments as well as all segments created in the
              future (until this file is reset to 0).  Note that existing
              segments that are attached to no process will be immediately
              destroyed when this file is set to 1.  Setting this option
              will also destroy segments that were created, but never
              attached, upon termination of the process that created the
              segment with shmget(2).

              Setting this file to 1 provides a way of ensuring that all
              System V shared memory segments are counted against the
              resource usage and resource limits (see the description of
              RLIMIT_AS in getrlimit(2)) of at least one process.

              Because setting this file to 1 produces behavior that is non‐
              standard and could also break existing applications, the
              default value in this file is 0.  Set this file to 1 only if
              you have a good understanding of the semantics of the applica‐
              tions using System V shared memory on your system.

       /proc/sys/kernel/shmall (since Linux 2.2)
              This file contains the system-wide limit on the total number
              of pages of System V shared memory.

       /proc/sys/kernel/shmmax (since Linux 2.2)
              This file can be used to query and set the run-time limit on
              the maximum (System V IPC) shared memory segment size that can
              be created.  Shared memory segments up to 1GB are now sup‐
              ported in the kernel.  This value defaults to SHMMAX.

       /proc/sys/kernel/shmmni (since Linux 2.4)
              This file specifies the system-wide maximum number of System V
              shared memory segments that can be created.

       /proc/sys/kernel/sysctl_writes_strict (since Linux 3.16)
              The value in this file determines how the file offset affects
              the behavior of updating entries in files under /proc/sys.
              The file has three possible values:

              -1  This provides legacy handling, with no printk warnings.
                  Each write(2) must fully contain the value to be written,
                  and multiple writes on the same file descriptor will over‐
                  write the entire value, regardless of the file position.

              0   (default) This provides the same behavior as for -1, but
                  printk warnings are written for processes that perform
                  writes when the file offset is not 0.

              1   Respect the file offset when writing strings into
                  /proc/sys files.  Multiple writes will append to the value
                  buffer.  Anything written beyond the maximum length of the
                  value buffer will be ignored.  Writes to numeric /proc/sys
                  entries must always be at file offset 0 and the value must
                  be fully contained in the buffer provided to write(2).

       /proc/sys/kernel/sysrq
              This file controls the functions allowed to be invoked by the
              SysRq key.  By default, the file contains 1 meaning that every
              possible SysRq request is allowed (in older kernel versions,
              SysRq was disabled by default, and you were required to
              specifically enable it at run-time, but this is not the case
              any more).  Possible values in this file are:

              0    Disable sysrq completely

              1    Enable all functions of sysrq

              > 1  Bit mask of allowed sysrq functions, as follows:
                     2  Enable control of console logging level
                     4  Enable control of keyboard (SAK, unraw)
                     8  Enable debugging dumps of processes etc.
                    16  Enable sync command
                    32  Enable remount read-only
                    64  Enable signaling of processes (term, kill, oom-kill)
                   128  Allow reboot/poweroff
                   256  Allow nicing of all real-time tasks

              This file is present only if the CONFIG_MAGIC_SYSRQ kernel
              configuration option is enabled.  For further details see the
              Linux kernel source file Documentation/admin-guide/sysrq.rst
              (or Documentation/sysrq.txt before Linux 4.10).

       /proc/sys/kernel/version
              This file contains a string such as:

                  #5 Wed Feb 25 21:49:24 MET 1998

              The "#5" means that this is the fifth kernel built from this
              source base and the date following it indicates the time the
              kernel was built.

       /proc/sys/kernel/threads-max (since Linux 2.3.11)
              This file specifies the system-wide limit on the number of
              threads (tasks) that can be created on the system.

              Since Linux 4.1, the value that can be written to threads-max
              is bounded.  The minimum value that can be written is 20.  The
              maximum value that can be written is given by the constant
              FUTEX_TID_MASK (0x3fffffff).  If a value outside of this range
              is written to threads-max, the error EINVAL occurs.

              The value written is checked against the available RAM pages.
              If the thread structures would occupy too much (more than
              1/8th) of the available RAM pages, threads-max is reduced
              accordingly.

       /proc/sys/kernel/yama/ptrace_scope (since Linux 3.5)
              See ptrace(2).

       /proc/sys/kernel/zero-paged (PowerPC only)
              This file contains a flag.  When enabled (nonzero), Linux-PPC
              will pre-zero pages in the idle loop, possibly speeding up
              get_free_pages.

       /proc/sys/net
              This directory contains networking stuff.  Explanations for
              some of the files under this directory can be found in tcp(7)
              and ip(7).

       /proc/sys/net/core/bpf_jit_enable
              See bpf(2).

       /proc/sys/net/core/somaxconn
              This file defines a ceiling value for the backlog argument of
              listen(2); see the listen(2) manual page for details.

       /proc/sys/proc
              This directory may be empty.

       /proc/sys/sunrpc
              This directory supports Sun remote procedure call for network
              filesystem (NFS).  On some systems, it is not present.

       /proc/sys/user (since Linux 4.9)
              See namespaces(7).

       /proc/sys/vm
              This directory contains files for memory management tuning,
              buffer and cache management.

       /proc/sys/vm/admin_reserve_kbytes (since Linux 3.10)
              This file defines the amount of free memory (in KiB) on the
              system that should be reserved for users with the capability
              CAP_SYS_ADMIN.

              The default value in this file is the minimum of [3% of free
              pages, 8MiB] expressed as KiB.  The default is intended to
              provide enough for the superuser to log in and kill a process,
              if necessary, under the default overcommit 'guess' mode (i.e.,
              0 in /proc/sys/vm/overcommit_memory).

              Systems running in "overcommit never" mode (i.e., 2 in
              /proc/sys/vm/overcommit_memory) should increase the value in
              this file to account for the full virtual memory size of the
              programs used to recover (e.g., login(1) ssh(1), and top(1))
              Otherwise, the superuser may not be able to log in to recover
              the system.  For example, on x86-64 a suitable value is 131072
              (128MiB reserved).

              Changing the value in this file takes effect whenever an
              application requests memory.

       /proc/sys/vm/compact_memory (since Linux 2.6.35)
              When 1 is written to this file, all zones are compacted such
              that free memory is available in contiguous blocks where pos‐
              sible.  The effect of this action can be seen by examining
              /proc/buddyinfo.

              Present only if the kernel was configured with CONFIG_COM‐
              PACTION.

       /proc/sys/vm/drop_caches (since Linux 2.6.16)
              Writing to this file causes the kernel to drop clean caches,
              dentries, and inodes from memory, causing that memory to
              become free.  This can be useful for memory management testing
              and performing reproducible filesystem benchmarks.  Because
              writing to this file causes the benefits of caching to be
              lost, it can degrade overall system performance.

              To free pagecache, use:

                  echo 1 > /proc/sys/vm/drop_caches

              To free dentries and inodes, use:

                  echo 2 > /proc/sys/vm/drop_caches

              To free pagecache, dentries and inodes, use:

                  echo 3 > /proc/sys/vm/drop_caches

              Because writing to this file is a nondestructive operation and
              dirty objects are not freeable, the user should run sync(1)
              first.

       /proc/sys/vm/legacy_va_layout (since Linux 2.6.9)
              If nonzero, this disables the new 32-bit memory-mapping lay‐
              out; the kernel will use the legacy (2.4) layout for all pro‐
              cesses.

       /proc/sys/vm/memory_failure_early_kill (since Linux 2.6.32)
              Control how to kill processes when an uncorrected memory error
              (typically a 2-bit error in a memory module) that cannot be
              handled by the kernel is detected in the background by hard‐
              ware.  In some cases (like the page still having a valid copy
              on disk), the kernel will handle the failure transparently
              without affecting any applications.  But if there is no other
              up-to-date copy of the data, it will kill processes to prevent
              any data corruptions from propagating.

              The file has one of the following values:

              1:  Kill all processes that have the corrupted-and-not-reload‐
                  able page mapped as soon as the corruption is detected.
                  Note that this is not supported for a few types of pages,
                  such as kernel internally allocated data or the swap
                  cache, but works for the majority of user pages.

              0:  Unmap the corrupted page from all processes and kill a
                  process only if it tries to access the page.

              The kill is performed using a SIGBUS signal with si_code set
              to BUS_MCEERR_AO.  Processes can handle this if they want to;
              see sigaction(2) for more details.

              This feature is active only on architectures/platforms with
              advanced machine check handling and depends on the hardware
              capabilities.

              Applications can override the memory_failure_early_kill set‐
              ting individually with the prctl(2) PR_MCE_KILL operation.

              Present only if the kernel was configured with CONFIG_MEM‐
              ORY_FAILURE.

       /proc/sys/vm/memory_failure_recovery (since Linux 2.6.32)
              Enable memory failure recovery (when supported by the plat‐
              form)

              1:  Attempt recovery.

              0:  Always panic on a memory failure.

              Present only if the kernel was configured with CONFIG_MEM‐
              ORY_FAILURE.

       /proc/sys/vm/oom_dump_tasks (since Linux 2.6.25)
              Enables a system-wide task dump (excluding kernel threads) to
              be produced when the kernel performs an OOM-killing.  The dump
              includes the following information for each task (thread,
              process): thread ID, real user ID, thread group ID (process
              ID), virtual memory size, resident set size, the CPU that the
              task is scheduled on, oom_adj score (see the description of
              /proc/[pid]/oom_adj), and command name.  This is helpful to
              determine why the OOM-killer was invoked and to identify the
              rogue task that caused it.

              If this contains the value zero, this information is sup‐
              pressed.  On very large systems with thousands of tasks, it
              may not be feasible to dump the memory state information for
              each one.  Such systems should not be forced to incur a per‐
              formance penalty in OOM situations when the information may
              not be desired.

              If this is set to nonzero, this information is shown whenever
              the OOM-killer actually kills a memory-hogging task.

              The default value is 0.

       /proc/sys/vm/oom_kill_allocating_task (since Linux 2.6.24)
              This enables or disables killing the OOM-triggering task in
              out-of-memory situations.

              If this is set to zero, the OOM-killer will scan through the
              entire tasklist and select a task based on heuristics to kill.
              This normally selects a rogue memory-hogging task that frees
              up a large amount of memory when killed.

              If this is set to nonzero, the OOM-killer simply kills the
              task that triggered the out-of-memory condition.  This avoids
              a possibly expensive tasklist scan.

              If /proc/sys/vm/panic_on_oom is nonzero, it takes precedence
              over whatever value is used in /proc/sys/vm/oom_kill_allocat‐
              ing_task.

              The default value is 0.

       /proc/sys/vm/overcommit_kbytes (since Linux 3.14)
              This writable file provides an alternative to
              /proc/sys/vm/overcommit_ratio for controlling the CommitLimit
              when /proc/sys/vm/overcommit_memory has the value 2.  It
              allows the amount of memory overcommitting to be specified as
              an absolute value (in kB), rather than as a percentage, as is
              done with overcommit_ratio.  This allows for finer-grained
              control of CommitLimit on systems with extremely large memory
              sizes.

              Only one of overcommit_kbytes or overcommit_ratio can have an
              effect: if overcommit_kbytes has a nonzero value, then it is
              used to calculate CommitLimit, otherwise overcommit_ratio is
              used.  Writing a value to either of these files causes the
              value in the other file to be set to zero.

       /proc/sys/vm/overcommit_memory
              This file contains the kernel virtual memory accounting mode.
              Values are:

                     0: heuristic overcommit (this is the default)
                     1: always overcommit, never check
                     2: always check, never overcommit

              In mode 0, calls of mmap(2) with MAP_NORESERVE are not
              checked, and the default check is very weak, leading to the
              risk of getting a process "OOM-killed".

              In mode 1, the kernel pretends there is always enough memory,
              until memory actually runs out.  One use case for this mode is
              scientific computing applications that employ large sparse
              arrays.  In Linux kernel versions before 2.6.0, any nonzero
              value implies mode 1.

              In mode 2 (available since Linux 2.6), the total virtual
              address space that can be allocated (CommitLimit in /proc/mem‐
              info) is calculated as

                  CommitLimit = (total_RAM - total_huge_TLB) *
                                overcommit_ratio / 100 + total_swap

              where:

                   *  total_RAM is the total amount of RAM on the system;

                   *  total_huge_TLB is the amount of memory set aside for
                      huge pages;

                   *  overcommit_ratio is the value in /proc/sys/vm/overcom‐
                      mit_ratio; and

                   *  total_swap is the amount of swap space.

              For example, on a system with 16GB of physical RAM, 16GB of
              swap, no space dedicated to huge pages, and an overcom‐
              mit_ratio of 50, this formula yields a CommitLimit of 24GB.

              Since Linux 3.14, if the value in /proc/sys/vm/overcom‐
              mit_kbytes is nonzero, then CommitLimit is instead calculated
              as:

                  CommitLimit = overcommit_kbytes + total_swap

              See also the description of /proc/sys/vm/admiin_reserve_kbytes
              and /proc/sys/vm/user_reserve_kbytes.

       /proc/sys/vm/overcommit_ratio (since Linux 2.6.0)
              This writable file defines a percentage by which memory can be
              overcommitted.  The default value in the file is 50.  See the
              description of /proc/sys/vm/overcommit_memory.

       /proc/sys/vm/panic_on_oom (since Linux 2.6.18)
              This enables or disables a kernel panic in an out-of-memory
              situation.

              If this file is set to the value 0, the kernel's OOM-killer
              will kill some rogue process.  Usually, the OOM-killer is able
              to kill a rogue process and the system will survive.

              If this file is set to the value 1, then the kernel normally
              panics when out-of-memory happens.  However, if a process lim‐
              its allocations to certain nodes using memory policies
              (mbind(2) MPOL_BIND) or cpusets (cpuset(7)) and those nodes
              reach memory exhaustion status, one process may be killed by
              the OOM-killer.  No panic occurs in this case: because other
              nodes' memory may be free, this means the system as a whole
              may not have reached an out-of-memory situation yet.

              If this file is set to the value 2, the kernel always panics
              when an out-of-memory condition occurs.

              The default value is 0.  1 and 2 are for failover of cluster‐
              ing.  Select either according to your policy of failover.

       /proc/sys/vm/swappiness
              The value in this file controls how aggressively the kernel
              will swap memory pages.  Higher values increase aggressive‐
              ness, lower values decrease aggressiveness.  The default value
              is 60.

       /proc/sys/vm/user_reserve_kbytes (since Linux 3.10)
              Specifies an amount of memory (in KiB) to reserve for user
              processes, This is intended to prevent a user from starting a
              single memory hogging process, such that they cannot recover
              (kill the hog).  The value in this file has an effect only
              when /proc/sys/vm/overcommit_memory is set to 2 ("overcommit
              never" mode).  In this case, the system reserves an amount of
              memory that is the minimum of [3% of current process size,
              user_reserve_kbytes].

              The default value in this file is the minimum of [3% of free
              pages, 128MiB] expressed as KiB.

              If the value in this file is set to zero, then a user will be
              allowed to allocate all free memory with a single process
              (minus the amount reserved by
              /proc/sys/vm/admin_reserve_kbytes).  Any subsequent attempts
              to execute a command will result in "fork: Cannot allocate
              memory".

              Changing the value in this file takes effect whenever an
              application requests memory.

       /proc/sysrq-trigger (since Linux 2.4.21)
              Writing a character to this file triggers the same SysRq func‐
              tion as typing ALT-SysRq-<character> (see the description of
              /proc/sys/kernel/sysrq).  This file is normally writable only
              by root.  For further details see the Linux kernel source file
              Documentation/admin-guide/sysrq.rst (or Documenta‐
              tion/sysrq.txt before Linux 4.10).

       /proc/sysvipc
              Subdirectory containing the pseudo-files msg, sem and shm.
              These files list the System V Interprocess Communication (IPC)
              objects (respectively: message queues, semaphores, and shared
              memory) that currently exist on the system, providing similar
              information to that available via ipcs(1).  These files have
              headers and are formatted (one IPC object per line) for easy
              understanding.  svipc(7) provides further background on the
              information shown by these files.

       /proc/thread-self (since Linux 3.17)
              This directory refers to the thread accessing the /proc
              filesystem, and is identical to the /proc/self/task/[tid]
              directory named by the process thread ID ([tid]) of the same
              thread.

       /proc/timer_list (since Linux 2.6.21)
              This read-only file exposes a list of all currently pending
              (high-resolution) timers, all clock-event sources, and their
              parameters in a human-readable form.

       /proc/timer_stats (from  Linux 2.6.21 until Linux 4.10)
              This is a debugging facility to make timer (ab)use in a Linux
              system visible to kernel and user-space developers.  It can be
              used by kernel and user-space developers to verify that their
              code does not make undue use of timers.  The goal is to avoid
              unnecessary wakeups, thereby optimizing power consumption.

              If enabled in the kernel (CONFIG_TIMER_STATS), but not used,
              it has almost zero run-time overhead and a relatively small
              data-structure overhead.  Even if collection is enabled at run
              time, overhead is low: all the locking is per-CPU and lookup
              is hashed.

              The /proc/timer_stats file is used both to control sampling
              facility and to read out the sampled information.

              The timer_stats functionality is inactive on bootup.  A sam‐
              pling period can be started using the following command:

                  # echo 1 > /proc/timer_stats

              The following command stops a sampling period:

                  # echo 0 > /proc/timer_stats

              The statistics can be retrieved by:

                  $ cat /proc/timer_stats

              While sampling is enabled, each readout from /proc/timer_stats
              will see newly updated statistics.  Once sampling is disabled,
              the sampled information is kept until a new sample period is
              started.  This allows multiple readouts.

              Sample output from /proc/timer_stats:

    $ cat /proc/timer_stats
    Timer Stats Version: v0.3
    Sample period: 1.764 s
    Collection: active
      255,     0 swapper/3        hrtimer_start_range_ns (tick_sched_timer)
       71,     0 swapper/1        hrtimer_start_range_ns (tick_sched_timer)
       58,     0 swapper/0        hrtimer_start_range_ns (tick_sched_timer)
        4,  1694 gnome-shell      mod_delayed_work_on (delayed_work_timer_fn)
       17,     7 rcu_sched        rcu_gp_kthread (process_timeout)
    ...
        1,  4911 kworker/u16:0    mod_delayed_work_on (delayed_work_timer_fn)
       1D,  2522 kworker/0:0      queue_delayed_work_on (delayed_work_timer_fn)
    1029 total events, 583.333 events/sec

              The output columns are:

              *  a count of the number of events, optionally (since Linux
                 2.6.23) followed by the letter 'D' if this is a deferrable
                 timer;

              *  the PID of the process that initialized the timer;

              *  the name of the process that initialized the timer;

              *  the function where the timer was initialized; and

              *  (in parentheses) the callback function that is associated
                 with the timer.

              During the Linux 4.11 development cycle, this file  was
              removed because of security concerns, as it exposes informa‐
              tion across namespaces.  Furthermore, it is possible to obtain
              the same information via in-kernel tracing facilities such as
              ftrace.

       /proc/tty
              Subdirectory containing the pseudo-files and subdirectories
              for tty drivers and line disciplines.

       /proc/uptime
              This file contains two numbers: the uptime of the system (sec‐
              onds), and the amount of time spent in idle process (seconds).

       /proc/version
              This string identifies the kernel version that is currently
              running.  It includes the contents of /proc/sys/kernel/ostype,
              /proc/sys/kernel/osrelease and /proc/sys/kernel/version.  For
              example:

        Linux version 1.0.9 (quinlan@phaze) #1 Sat May 14 01:51:54 EDT 1994

       /proc/vmstat (since Linux 2.6.0)
              This file displays various virtual memory statistics.  Each
              line of this file contains a single name-value pair, delimited
              by white space.  Some lines are present only if the kernel was
              configured with suitable options.  (In some cases, the options
              required for particular files have changed across kernel ver‐
              sions, so they are not listed here.  Details can be found by
              consulting the kernel source code.)  The following fields may
              be present:

              nr_free_pages (since Linux 2.6.31)

              nr_alloc_batch (since Linux 3.12)

              nr_inactive_anon (since Linux 2.6.28)

              nr_active_anon (since Linux 2.6.28)

              nr_inactive_file (since Linux 2.6.28)

              nr_active_file (since Linux 2.6.28)

              nr_unevictable (since Linux 2.6.28)

              nr_mlock (since Linux 2.6.28)

              nr_anon_pages (since Linux 2.6.18)

              nr_mapped (since Linux 2.6.0)

              nr_file_pages (since Linux 2.6.18)

              nr_dirty (since Linux 2.6.0)

              nr_writeback (since Linux 2.6.0)

              nr_slab_reclaimable (since Linux 2.6.19)

              nr_slab_unreclaimable (since Linux 2.6.19)

              nr_page_table_pages (since Linux 2.6.0)

              nr_kernel_stack (since Linux 2.6.32)
                     Amount of memory allocated to kernel stacks.

              nr_unstable (since Linux 2.6.0)

              nr_bounce (since Linux 2.6.12)

              nr_vmscan_write (since Linux 2.6.19)

              nr_vmscan_immediate_reclaim (since Linux 3.2)

              nr_writeback_temp (since Linux 2.6.26)

              nr_isolated_anon (since Linux 2.6.32)

              nr_isolated_file (since Linux 2.6.32)

              nr_shmem (since Linux 2.6.32)
                     Pages used by shmem and tmpfs(5).

              nr_dirtied (since Linux 2.6.37)

              nr_written (since Linux 2.6.37)

              nr_pages_scanned (since Linux 3.17)

              numa_hit (since Linux 2.6.18)

              numa_miss (since Linux 2.6.18)

              numa_foreign (since Linux 2.6.18)

              numa_interleave (since Linux 2.6.18)

              numa_local (since Linux 2.6.18)

              numa_other (since Linux 2.6.18)

              workingset_refault (since Linux 3.15)

              workingset_activate (since Linux 3.15)

              workingset_nodereclaim (since Linux 3.15)

              nr_anon_transparent_hugepages (since Linux 2.6.38)

              nr_free_cma (since Linux 3.7)
                     Number of free CMA (Contiguous Memory Allocator) pages.

              nr_dirty_threshold (since Linux 2.6.37)

              nr_dirty_background_threshold (since Linux 2.6.37)

              pgpgin (since Linux 2.6.0)

              pgpgout (since Linux 2.6.0)

              pswpin (since Linux 2.6.0)

              pswpout (since Linux 2.6.0)

              pgalloc_dma (since Linux 2.6.5)

              pgalloc_dma32 (since Linux 2.6.16)

              pgalloc_normal (since Linux 2.6.5)

              pgalloc_high (since Linux 2.6.5)

              pgalloc_movable (since Linux 2.6.23)

              pgfree (since Linux 2.6.0)

              pgactivate (since Linux 2.6.0)

              pgdeactivate (since Linux 2.6.0)

              pgfault (since Linux 2.6.0)

              pgmajfault (since Linux 2.6.0)

              pgrefill_dma (since Linux 2.6.5)

              pgrefill_dma32 (since Linux 2.6.16)

              pgrefill_normal (since Linux 2.6.5)

              pgrefill_high (since Linux 2.6.5)

              pgrefill_movable (since Linux 2.6.23)

              pgsteal_kswapd_dma (since Linux 3.4)

              pgsteal_kswapd_dma32 (since Linux 3.4)

              pgsteal_kswapd_normal (since Linux 3.4)

              pgsteal_kswapd_high (since Linux 3.4)

              pgsteal_kswapd_movable (since Linux 3.4)

              pgsteal_direct_dma

              pgsteal_direct_dma32 (since Linux 3.4)

              pgsteal_direct_normal (since Linux 3.4)

              pgsteal_direct_high (since Linux 3.4)

              pgsteal_direct_movable (since Linux 2.6.23)

              pgscan_kswapd_dma

              pgscan_kswapd_dma32 (since Linux 2.6.16)

              pgscan_kswapd_normal (since Linux 2.6.5)

              pgscan_kswapd_high

              pgscan_kswapd_movable (since Linux 2.6.23)

              pgscan_direct_dma

              pgscan_direct_dma32 (since Linux 2.6.16)

              pgscan_direct_normal

              pgscan_direct_high

              pgscan_direct_movable (since Linux 2.6.23)

              pgscan_direct_throttle (since Linux 3.6)

              zone_reclaim_failed (since linux 2.6.31)

              pginodesteal (since linux 2.6.0)

              slabs_scanned (since linux 2.6.5)

              kswapd_inodesteal (since linux 2.6.0)

              kswapd_low_wmark_hit_quickly (since 2.6.33)

              kswapd_high_wmark_hit_quickly (since 2.6.33)

              pageoutrun (since Linux 2.6.0)

              allocstall (since Linux 2.6.0)

              pgrotated (since Linux 2.6.0)

              drop_pagecache (since Linux 3.15)

              drop_slab (since Linux 3.15)

              numa_pte_updates (since Linux 3.8)

              numa_huge_pte_updates (since Linux 3.13)

              numa_hint_faults (since Linux 3.8)

              numa_hint_faults_local (since Linux 3.8)

              numa_pages_migrated (since Linux 3.8)

              pgmigrate_success (since Linux 3.8)

              pgmigrate_fail (since Linux 3.8)

              compact_migrate_scanned (since Linux 3.8)

              compact_free_scanned (since Linux 3.8)

              compact_isolated (since Linux 3.8)

              compact_stall (since Linux 2.6.35)
                     See the kernel source file Documentation/vm/tran‐
                     shuge.txt.

              compact_fail (since Linux 2.6.35)
                     See the kernel source file Documentation/vm/tran‐
                     shuge.txt.

              compact_success (since Linux 2.6.35)
                     See the kernel source file Documentation/vm/tran‐
                     shuge.txt.

              htlb_buddy_alloc_success (since Linux 2.6.26)

              htlb_buddy_alloc_fail (since Linux 2.6.26)

              unevictable_pgs_culled (since Linux 2.6.28)

              unevictable_pgs_scanned (since Linux 2.6.28)

              unevictable_pgs_rescued (since Linux 2.6.28)

              unevictable_pgs_mlocked (since Linux 2.6.28)

              unevictable_pgs_munlocked (since Linux 2.6.28)

              unevictable_pgs_cleared (since Linux 2.6.28)

              unevictable_pgs_stranded (since Linux 2.6.28)

              thp_fault_alloc (since Linux 2.6.39)
                     See the kernel source file Documentation/vm/tran‐
                     shuge.txt.

              thp_fault_fallback (since Linux 2.6.39)
                     See the kernel source file Documentation/vm/tran‐
                     shuge.txt.

              thp_collapse_alloc (since Linux 2.6.39)
                     See the kernel source file Documentation/vm/tran‐
                     shuge.txt.

              thp_collapse_alloc_failed (since Linux 2.6.39)
                     See the kernel source file Documentation/vm/tran‐
                     shuge.txt.

              thp_split (since Linux 2.6.39)
                     See the kernel source file Documentation/vm/tran‐
                     shuge.txt.

              thp_zero_page_alloc (since Linux 3.8)
                     See the kernel source file Documentation/vm/tran‐
                     shuge.txt.

              thp_zero_page_alloc_failed (since Linux 3.8)
                     See the kernel source file Documentation/vm/tran‐
                     shuge.txt.

              balloon_inflate (since Linux 3.18)

              balloon_deflate (since Linux 3.18)

              balloon_migrate (since Linux 3.18)

              nr_tlb_remote_flush (since Linux 3.12)

              nr_tlb_remote_flush_received (since Linux 3.12)

              nr_tlb_local_flush_all (since Linux 3.12)

              nr_tlb_local_flush_one (since Linux 3.12)

              vmacache_find_calls (since Linux 3.16)

              vmacache_find_hits (since Linux 3.16)

              vmacache_full_flushes (since Linux 3.19)

       /proc/zoneinfo (since Linux 2.6.13)
              This file display information about memory zones.  This is
              useful for analyzing virtual memory behavior.
\end{verbatim}

\subsection{\texorpdfstring{\protect\hyperlink{NOTES}{}NOTES ~ ~ ~ ~
\protect\hyperlink{top_of_page}{{top}}}{NOTES ~ ~ ~ ~ top}}\label{notes-top}

\begin{verbatim}
       Many strings (i.e., the environment and command line) are in the
       internal format, with subfields terminated by null bytes ('\0'), so
       you may find that things are more readable if you use od -c or tr
       "\000" "\n" to read them.  Alternatively, echo `cat <file>` works
       well.

       This manual page is incomplete, possibly inaccurate, and is the kind
       of thing that needs to be updated very often.
\end{verbatim}

\subsection{\texorpdfstring{\protect\hyperlink{SEE_ALSO}{}SEE ALSO ~ ~ ~
~
\protect\hyperlink{top_of_page}{{top}}}{SEE ALSO ~ ~ ~ ~ top}}\label{see-also-top}

\begin{verbatim}
       cat(1), dmesg(1), find(1), free(1), init(1), ps(1), tr(1), uptime(1),
       chroot(2), mmap(2), readlink(2), syslog(2), slabinfo(5), sysfs(5),
       hier(7), namespaces(7), time(7), arp(8), hdparm(8), ifconfig(8),
       lsmod(8), lspci(8), mount(8), netstat(8), procinfo(8), route(8),
       sysctl(8)

       The Linux kernel source files: Documentation/filesystems/proc.txt
       Documentation/sysctl/fs.txt, Documentation/sysctl/kernel.txt,
       Documentation/sysctl/net.txt, and Documentation/sysctl/vm.txt.
\end{verbatim}

\subsection{\texorpdfstring{\protect\hyperlink{COLOPHON}{}COLOPHON ~ ~ ~
~
\protect\hyperlink{top_of_page}{{top}}}{COLOPHON ~ ~ ~ ~ top}}\label{colophon-top}

\begin{verbatim}
       This page is part of release 4.16 of the Linux man-pages project.  A
       description of the project, information about reporting bugs, and the
       latest version of this page, can be found at
       https://www.kernel.org/doc/man-pages/.

Linux                            2017-09-15                          PROC(5)
\end{verbatim}
\end{document}
